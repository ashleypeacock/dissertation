\documentclass[11pt]{report}

\usepackage{graphicx}
\usepackage{fullpage}

\newcommand{\define}[2] {
  \textbf{Definition: #1}
  \begin{center} #2
\end{center}
}

\begin{document}

\section{Question 1}
\subsection{Part A}
Joint distribution is: $P(D)P(S)P(C|D)P(I|C)P(V|C,S,J)P(J|D)P(T|J)$

\subsection{Part B}

\subsection{Part C}
Found this question a bit confusing so I'll explain best I can...

P(S|J) = J $\rightarrow$ V $\leftarrow$ (no blocking)
P(S|D) = D $\rightarrow$ C $\rightarrow$ V $\leftarrow$ S (no blocking)
P(S|I) = I $\leftarrow$ C $\rightarrow$ V $\leftarrow$ S (no blocking)
P(S|C) = C $\rightarrow$ V $\leftarrow$ S (no blocking)

For paths C $\rightarrow$ V $\leftarrow$ S and J $\rightarrow$ V $\leftarrow$ S both paths are always unblocked as V is conditioned on.

By conditioning on the variables J, D, I, C, no blocking occurs and hence S is not conditionally independent on J, D, I or C and there is therefore no observation that would lead to no extra information(all would lead to an increase in information).

If you could condition on both C and J, S would become conditionally independent of D but as we are limited to one additional variable, if you condition simply on C or J, there is still a path from D to S via the symmetric route and as such D is not independent. 


\end{document}