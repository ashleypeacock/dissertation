\documentclass[11pt]{report}
\begin{document}

\title{Autism Simulator}
\author{Ashley Peacock}
\maketitle
\tableofcontents
\chapter{Introduction}

\section{Selecting a project}


\chapter{Literature review}
\section{What is Autism?}
Autism is a spectrum disorder that affects social communication, social imagination and social interaction. The range and severity of difficulties will vary for each individual. For example those with Asperger's syndrome may have more problems with socialising whereas those with Low-functioning autism may only speak single words or short sentences. Aspergers is therefore looked upon as a 'hidden condition', making it even harder for the general public to understand. 
In addition to these issues, often those with autism experience Sensory Processing Disorder. 
\subsection{Triad of Impairments}

\subsection{Other difficulties}
\subsubsection{Sensory overloads}
\subsubsection{Meltdowns}
\subsubsection{Fears}
\section{Impact and Prevalence}
Although it is difficult to calculate an exact figure, it is estimated in the UK that there is approximately 1.1\% of the 
population with Autism, based on the 2011 census\cite{nas}. This figure however appears to be rising across the globe as awareness and understanding of the condition increases alongside broadening criteria\cite{increasingprevalence}, for example the inclusion of Aspergers syndrome which has only been a formal diagnosis since 1990. Early estimates of autism spectrum disorders were identified as less than 10 in 10,000, however this has grown to a current estimation of 110 in 10,000 in the USA \cite{increasingprevalence}. Increasing prevalence has been found across various countries in the world.

It is estimated that only 22\% of teachers have been trained specifically in autism and the majority of training given is typically one to four hours. 54\% of all teachers in England do not feel they have had adequate training to teach children with autism.\cite{statsandfacts} 30\% of parents of children with autism in mainstream education are satisfied with the level of understanding of autism across the school\cite{nasschool}. 23\% of parents are dissatisfied with SENCO's level of understanding of autism. 

It is estimated that around 40\% of children with autism have been bullied at school. 1 in 5 children with autism have been excluded from school \cite{nasschool}. 24.4\% of pupils with autism achieved 5A*-C GCSEs in 2010/2011 in comparison to 58.2\% of the population\cite{statsandfacts}, a surprising figure owing that people with autism are deemed to have average to above average intelligence\cite{findreference}.

\begin{quote}
It doesn't appear that mainstream teachers have had access to training. The fundamental issues relating to communication, behaviour and language disorder continue to be misinterpreted as 'bad behaviour', 'not listening' and so on.\cite{nasschool}
\end{quote}

\begin{quote}
Danny would not have been excluded if the school had understood the difference between 'normal' behaviour and Aspergers syndrome. They inflamed situations because they didn't understand that my son finds physical contact, or being touched by teachers, really difficult \cite{nasschool}
\end{quote}

\begin{quote}
Action to increase understanding of autism across the whole school and to provide support with social activities can make a huge difference to whether a child with autism feels included at school.\cite{nasschool}
\end{quote}

\begin{quote}
If I could make one change...I would ensure compulsory, thorough training about autism and how it affects learning is given to all school staff. Plus, ongoing support for all schools from well-resourced autism advisory teams at local authority level\cite{nasschool}
\end{quote}

\begin{quote}
If I could make one change... every person who comes into contact with my daughter would have some form of training in autism.\cite{nasschool}
\end{quote}


\section{Public perception}
Overall although there is some level of awareness of autism in the public domain, there is still much left to be desired.  From a survey carried out by the National Autistic Society, 92\% of respondents had heard of Autism but only 48\% had heard of Aspergers syndrome which has less obvious difficulties and is consequently known as a 'hidden disability'. Most were able to identify key characteristics such as difficulty communicating or making friends whilst other common characteristics such as a need for 'clear unambiguous instructions, being disturbed by noise and touch and having difficulty sleeping', were less known. 


\section{Previous work}

\subsection{Education software}
How other education software can be used to help children with understanding autism. 
\subsection{Other simulators}

\chapter{Design process}
\section{Interviews}

\section{Game design}
\subsection{House design}
House design choices. 
\subsection{Character}
\subsubsection{Autism aspects to convey}
\subsection{Design of sensory system}
\subsection{Story boards}

\chapter{Prototype}

\section{Implementation}
What was in the prototype.
\section{Evaluation}
\subsection{Expert feedback}
\subsection{User feedback}
\section{Improvements}

\chapter{First version}

\chapter{Final version}

\begin{thebibliography}{9}

\bibitem{findref}
Find a reference

\bibitem{increasingprevalence}
Johnny L. Matson, Alison M. Kozlowski
The increasing prevalence of autism spectrum disorders. Research in Autism Spectrum Disorders(2011)

\bibitem{nas}
National autistic society. www.autism.org.uk

\bibitem{statsandfacts}
Ambitious about autism. www.ambitiousaboutautism

\bibitem{nasschool}
Make school make sense. Autism and education: the reality for families today. National Autistic Society, 

\bibitem{autismmisconception}
Autism misconceptions. NHS


\end{thebibliography}

\end{document}