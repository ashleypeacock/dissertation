\documentclass[11pt]{article}

\begin{document}

\section{Introduction}
Meta-cognition can be defined as our ability to "Know about knowing". It can be broken into three components: meta cognitive knowledge, meta-cognitive skill and meta-cognitive experiences. It influences our ability to learn and problem solve, regulate our emotions and motivations and therefore complete goals. Teaching meta-cognition will enable students to become better and more efficient future learners. 

A greater meta-cognition ability has shown to be a factor in predicting students learning outcomes. 

As a domain independent skill, meta-cognition can influence not only our ability to learn, but our ability to problem solve in all aspects of life.

As a domain independent skill, meta-cognition is important. Greater learning and understanding acquired about meta-cognition can be transferred to many areas of life, making tasks, problem solving and learning more efficient in comparison to domain specific learning goals which seldom be applied elsewhere. Teaching meta-cognitive skills to students will enable students to become more efficient and ideally, successful learners. 

"To what extent have tutoring systems been effective in teaching metacognition"

We'll start by looking at two systems: Cognitive tutors and Betty's brain.

\section{Paper 1: Cognitive tutors}

The paper adapts principles put forward by Anderson in an attempt to teach students meta-cognition. Anderson's principles have been used extensively and successfully in cognitive tutors and domain teaching and as such, should provide a solid platform to build upon for teaching meta-cognition. The paper designs and combines a help-model with a cognitive tutor in an effort to improve students help-seeking skills which should be transferred across domains. By increasing meta-cognitive skills, it is hoped that students become better learners across disciplines and not simply the domain studied. 

\subsection{Help Tutor}
The Help-tutor is an add-on to cognitive tutor and designed specifically to strengthen students help-seeking abilities. The Help-tutor traces a pre-defined model of desired student help-seeking and uses this to predict what actions it feel would be the most beneficial. When the user deviates from this model it is classified as one of the two help-seeking errors:

\begin{enumerate}
\item When the user fails or avoids help when required.
\item When the user abuses help, by for example, repeatedly asking for help.
\end{enumerate}

Effectively, the design is attempting to balance the user’s helping seeking, encouraging them to seek help when there is obvious difficulty, and discouraging the overuse of asking for help when difficulty arises. It encourages them to have a good attempt at the problem first, and then think about the problem and their knowledge in order to identify where and when they need to seek help, rather than 'gaming the system' to just find the correct answer.

The help-seeking model was designed iteratively, firstly formulated from research and literature and then improved following multiple sets of experiments. Minimal information is given on what improvements were made with each iteration. The Help-tutor was found to be successful in improving students help-seeking abilities whilst using the cognitive tutor, however, these skills were not transferred elsewhere and no improvement at domain-level learning was observed. 


\subsection{Discussion}
The paper fails to reach the goals it sets out to achieve: to improve students metacognition to enable them to be more successful learners in other domains and to better help students learning in the specific domain. 

\section{Paper 2: Bettys brain}

\subsection{Paper review}

\subsection{Discussion}

\subsection{Conclusion}

\begin{thebibliography}{9}

\bibitem{studenthelp}
Limitations of student control: Do students know when they need help?. In Intelligent Tutoring Systems (pp. 292-303). Aleven, V., & Koedinger, K. R

\bibitem{designmeta}
Designing for meta-cognition: applying cognitive tutor principles to the tutoring of help seeking. Meta-cognition and Learning, 2(2-3), 125-140. Roll, I., Aleven.V, McLaren B. M., Koedinger, K. R. (2007)

\bibitem{metachildren}
The Specificity of Developing Meta-cognition at Children with Learning Difficulties 

\bibitem{openended}
Modelling Learner’s Cognitive and Metacognitive Strategies in an Open-Ended Learning Environment. James R. Segedy, John S. Kinnebrew, Gautam Biswas

\bibitem{modelling}
Modelling and Measuring Self-Regulated Learning in Teachable Agent Environments. John S. Kinnebrew, Gautam Biswas


\end{thebibliography}
\end{document}