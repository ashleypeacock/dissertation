\documentclass[11pt]{article}

\begin{document}

\section{Introduction}
"To what extent have the proposed models succeeded in their goals of aiding or developing meta-cognition in students?"

\subsection{What is Metacognition?}
The topic of meta-cognition is as broad as it's vast applications, but it can be thought of as "Knowledge about knowledge" or "Thinking about thinking". Metacognitive skills include our ability to regulate our knowledge, identifying where gaps may be and employing suitable strategies to fill them. Certain meta-cognitive skills learnt in one domain can be easily transferred to another domain. For example, learning to ride a bicycle does not automatically enable me to ride a skateboard, however, knowledge on oneself and learning processes developed from learning to ride a bike may be useful in learning to ride a skateboard. For example, being aware that it may be suboptimal to continue learning after a period of time, or that it is more effective to learn how to ride when asking for help and to be given a push. 

Successful teaching of meta-cognition would show an enhancement in domain learning and ideally show a closer fit the proposed models. 

\subsection{Challenges involved in designing for Meta-cognition}
Designing for meta-cognition is a very challenging task. Owing these models are required to fit the many different paradigms of minds, it becomes clear why this is such a challenging task. 

In addition to internal challenges in designing and building for meta-cognition, there are additional external obstacles: the student themselves. The goal is to encourage students to think about their own thinking and learning, but as each person brings fourth a set of unique personality traits, experience and set of knowledge, it is very difficult to be able to define a universal model of learning, and then proceed to finding the best universal way to teach it. However, intelligent tutoring systems are required to use models as a means to monitor students and motivate them away from sub-optimal learning strategies(such as repeatedly asking for help) and as such, exploring the extent in which these goals have or have not been obtained is important. But, we cannot always model responses for external factors, for example, does the user have difficulty with reading or writing, preferring auditory or visual information? Is the user impatient? Does an ITS fit with all of the different learning styles? Does the user already have a a large amount of metacognitive knowledge and thus a system allowing more choice and freedom may better suit? Is the user simply fixed in their methods and approaching or unwilling to adopt alternative strategies?

In \cite{designmeta} it was explained that by adding to the cognitive load, the meta-cognitive goals became secondary to that of domain learning, hence, bringing meta-cognition to the forefront and attention of the user whilst in a domain-specific environment can remove its effectiveness and it is better kept as a subtle, rather than explicit addition. Else, students are required to be properly motivated to give meta-cognitive strategies the attention required. 

*** set it up here as to why help-seeking is therefore the easiest and best challenge to solve first. It's more universal than tacking other tasks.

\section{Cognitive tutors}

One of the best approaches in designing an ITS was to give the user as much control over their learning as possible, allowing students to request help when required as they are thought to be the better judge of their own needs. However, research has shown that students help-seeking skills are sub-optimal \cite{studenthelp}. Students were found to take too long to request hints and when they did take them, would ignore high-level hints and repeatedly ask for helps/hints until the solution complete. This represents one of two help-seeking errors, the other being the avoidance of help. 

\section{Betty's brain}


\begin{thebibliography}{9}

\bibitem{studenthelp}
Limitations of student control: Do students know when they need help?. In Intelligent Tutoring Systems (pp. 292-303). Aleven, V., Koedinger, K. R

\bibitem{designmeta}
Designing for meta-cognition: applying cognitive tutor principles to the tutoring of help seeking. Meta-cognition and Learning, 2(2-3), 125-140. Roll, I., Aleven.V, McLaren B. M., Koedinger, K. R. (2007)

\bibitem{metachildren}
The Specificity of Developing Meta-cognition at Children with Learning Difficulties 

\bibitem{openended}
Modelling Learner’s Cognitive and Metacognitive Strategies in an Open-Ended Learning Environment. James R. Segedy, John S. Kinnebrew, Gautam Biswas

\bibitem{modelling}
Modelling and Measuring Self-Regulated Learning in Teachable Agent Environments. John S. Kinnebrew, Gautam Biswas


\end{thebibliography}
\end{document}