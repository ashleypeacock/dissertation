\chapter{Appendix}

\chapter{Summative}
- vingettes

\chapter{Prototype implementation}

\section{HouseScene}
\begin{lstlisting}

public class HouseScene extends SceneManager {

    public HouseScene(String houseScene, Main app) {
        super(houseScene, app);
    }

    @Override
    public void setupModels() {
        setupLights();
        setupDoors();
        setupDescriptions();

    }
    
    
    public void setupLights() {
        Iterator it = models.keySet().iterator();
        while (it.hasNext()) {
            String key = it.next().toString();
            if(key.contains("lamp")) {
                Spatial node = models.get(key);
                addLampLight(node);
                node.setUserData("state", true);
                node.setUserData("sensory", "lamp");
                System.out.println("Added light to " + key);
                Main.getSensoryManager().addSensoryObject(node, "light");
            }
        }
        
    }
    
    public void setupDescriptions() {
        addDescription("fridge", "This is a fridge and it smells");
        addDescription("fryingpan", "Urg, the smell of this is overpowering");
        addDescription("wardrobe", "Certain clothes can be very uncomfortable for someone with Autism. One person describes"
                + "it as like wearing sand paper, and clothes labels feeling like sandpaper");
        addDescription("tv", "TV Adverts: Parents be warned. If you buy a toy from an advert and it doesn't do"
                + "exactly what the advert says, your child may be quite surprised and upset. Similarly, things on TV "
                + "can be taken very literally and a child with autism may not know they what they see is not real.");
        
    }
    
    public void setupDoors() {
      Iterator it = models.keySet().iterator();
        while (it.hasNext()) {
            String key = it.next().toString();
            if(!key.contains("door")) 
                continue; 
            
            Node door = (Node) models.get(key);
            if(door.getName().contains("door")) {
                DoorControl dc = new DoorControl();
                door.addControl(dc);               
            }
        }
    }

    @Override
    public void setupMaterials() {
        Material m = new Material(am, "Common/MatDefs/Misc/Unshaded.j3md");
        MaterialsManager.addMaterial("unshaded", m); 
    }
}

\end{lstlisting}

\chapter{First version}

\section{Rewrite of scene manager}

\subsection{HomeScene}
\begin{lstlisting}
/*
 * To change this template, choose Tools | Templates
 * and open the template in the editor.
 */
package mygame.scene.scenes;

import com.jme3.asset.AssetManager;
import com.jme3.audio.AudioNode;
import com.jme3.math.Vector3f;
import com.jme3.scene.Node;
import com.jme3.scene.Spatial;
import java.util.ArrayList;
import mygame.Main;
import mygame.MainGameAppState;
import mygame.events.GameActionEvent;
import mygame.events.GameActionListener;
import mygame.scene.MaterialsManager;
import mygame.scene.Scene;
import mygame.scene.SceneManager;
import mygame.sensory.LightSensoryObject;
import mygame.sensory.LightSensoryObject.LightType;
import mygame.sensory.SensoryObject;

/**
 * Contains all the information for the home scene.
 * @author Ashley
 */
public class HomeScene extends SceneManager implements GameActionListener {
    Scene kitchen;
    Scene livingroom;
    Scene bedroom;
    Scene bathroom;
    Scene hallway;
    
    AssetManager am = null;
    Main app;
    
    /** Bedroom sensory objects **/
    SensoryObject bedroom_alarmSO;
    LightSensoryObject bedroom_lampSO;
    LightSensoryObject bedroom_ceilingSO;
    
    /** Bathroom objects **/
    SensoryObject bathroom_toiletSO;
    SensoryObject bathroom_toothBrushSO; // this is a cup atm, change to toothrbush when I get one.
    SensoryObject bathroom_ceilingLampSO;
    
    /** Hallway objects **/
    LightSensoryObject hallwayLampSO;
    LightSensoryObject hallwayCeilingSO;
    
    /** Kitchen objects **/
    LightSensoryObject kitchenCeilingSO;
    SensoryObject washingMachineSO;
    AudioNode drinkWater = new AudioNode();
    LightSensoryObject fLightSO;
    
    /** Livingroom objects **/
    LightSensoryObject livingroomFloorLampSO;
    LightSensoryObject livingroomCeilingSO;
    SensoryObject livingroomHooverSO;
    
    public HomeScene(Main app) {
        super(app);
        
        
        this.am = app.getAssetManager();
        this.app = app; 
        
        bedroom = new Scene("bedroom", app);
        bedroom.setSceneLoadPoint(new Vector3f(3.9f, 0.45f, 4.2f), new Vector3f(-0.99f, -0.03f, -0.04f));
        bedroom.setScale(3f);
        bedroom.assignSpatialActions("wardrobe", "dressed");
        bedroom.assignSpatialThought("routine", "Routines make me feel safer!");
        bedroom.addDescription("routine", "People with autism often have a desire for a strict routine. One minute off can throw them off.");
               
        bathroom = new Scene("bathroom", app);
        bathroom.setSceneLoadPoint(new Vector3f(0.844f, 2.1f, -6.54f), new Vector3f(-0.01f, 0.04f, 0.99f));
        bathroom.setScale(4f);
        
        livingroom = new Scene("livingroom", app);
        livingroom.setSceneLoadPoint(new Vector3f(5, 4, 5), new Vector3f(-0.14f, -0.013f, -0.99f));
        livingroom.assignSpatialActions("mill.001", "play");
        livingroom.assignSpatialActions("mill.002", "play");
        livingroom.assignSpatialActions("millbase", "play");
        
        livingroom.addDescription("lightening_poster", "Thunderstorms was one of the top listed 'Unusual fears' for children with Autism. During interviews, two people with autism explained they could literally feel the air change before a storm. "
                + "This caused stressed for one and excitement for another. Sensory problems in Autism are unique for each individual"); 
        
        hallway = new Scene("hallway", app);
        hallway.setSceneLoadPoint(new Vector3f(-9.09f, 2.04f, 5.99f), new Vector3f(-0.99f, -0.37f, 0.02f));
        hallway.setScale(3f);
        
        kitchen = new Scene("kitchen", app);
        kitchen.setSceneLoadPoint(new Vector3f(7.42f, 1.25f, 2.36f), new Vector3f(0.02f, -0.05f, -0.99f));
        kitchen.setScale(2f);
        
        kitchen.assignSpatialActions("Fridge", "food");
        kitchen.assignSpatialActions("cheese", "food");
        kitchen.assignSpatialActions("grapes", "food");
        kitchen.assignSpatialActions("turkeyleg", "food");
        kitchen.assignSpatialActions("foodinbowl", "food");
        kitchen.assignSpatialActions("lamp", "sensory");
        
        setMeltdownRespawn(bedroom);
        loadScene(bedroom);        
        addGameActionListener(this);
        setupSensoryObjects();
        addHouseDescriptions();
    }
    
    public void addHouseDescriptions() {
           bedroom.assignSpatialThought("dinosaur", "Play with me");
           bedroom.assignSpatialThought("wardrobe", "Certain clothes can be very uncomfortable for someone with Autism. One person describes "
                   + "them as like wearing sand paper, and the labels on them feeling like barbed wire.");
           bedroom.assignSpatialThought("ceilinglamp", "Some of these lights hurt my eyes");
           
           livingroom.assignSpatialThought("tv", "Wow. I want a dinosaur as big as on TV! Can't wait to get him for christmas");
           livingroom.addDescription("tv", "Literal interpretation: children with autism tend to interpret information literally. If they recieve a toy seen on TV they may not realise it does not behave in the same way. i.e toy Buzz Lightyear won't fly in real life!");
           livingroom.assignSpatialThought("mill.001", "Oh oh oh. I like this!");
           livingroom.assignSpatialThought("mill.002", "Oh oh oh. I like this!");
           livingroom.assignSpatialThought("millbase", "Oh oh oh. I like this!");
    }

    /**
     * Handles all the object actions of the scene.
     * @param gae
     * @return 
     */
    public boolean notify(GameActionEvent gae) {
        String currentSceneName = currentScene.getName();
        String actionSpatial = gae.getSpatial().getName();
        Spatial actionParentSpatial = gae.getSpatial().getParent();
        
        // this may cause problems because it's getting the second level of nodes rather than
        // finding them.
        //System.out.println("Acted on spatial " + actionSpatial + " in room " + currentSceneName + ", ");
        
        if(currentSceneName.equals("kitchen")) {
            if(actionSpatial.contains("door")) {
                hallway.setSceneLoadPoint(new Vector3f(-5.18f, 2.04f, 6.9f), new Vector3f(-0.014f, -0.03f, -0.99f));
                changeScene(hallway);
            }else if(actionSpatial.contains("switch")) {
                kitchenCeilingSO.doAction();
                fLightSO.doAction();
            }
            //System.out.println("Action spatial " + actionSpatial);
        }else if(currentSceneName.equals("livingroom")) {
            if(actionSpatial.contains("door")) {
                hallway.setSceneLoadPoint(new Vector3f(3.32f, 1.9f, -4.2f), new Vector3f(-0.01f, -0.03f, 0.99f));
                changeScene(hallway);
            }else if(actionSpatial.contains("switch")) {
                livingroomCeilingSO.doAction();
            }
        }else if(currentSceneName.equals("bathroom")) {
            if(actionSpatial.equals("door")) {
                hallway.setSceneLoadPoint(new Vector3f(12.7f, 2.01f, 4.2f), new Vector3f(0.216f, -0.03f, 0.976f));
                changeScene(hallway);
            }else if(actionSpatial.contains("toilet")) {
                bathroom_toiletSO.doAction();
            }else if(actionSpatial.contains("switch")) {
                bathroom_ceilingLampSO.doAction();
            }
            
        } else if(currentSceneName.equals("bedroom")) {
             if(actionSpatial.equals("door")) {
                 hallway.setSceneLoadPoint(new Vector3f(21.9f, 1.9f, 5.7f), new Vector3f(-0.99f, -0.37f, 0.02f));
                changeScene(hallway);             
             } else if(actionSpatial.contains("clock") || actionParentSpatial.getName().contains("clock")) {
                 System.out.println("Acting on clock");
                 bedroom_alarmSO.doAction();               
             } 
             
             if(actionSpatial.contains("switch")) {
                 bedroom_ceilingSO.doAction();
             }
             
        } else if(currentSceneName.equals("hallway")) {
            if(actionSpatial.contains("door")) {
                String actionSpatialID = actionParentSpatial.getParent().getName(); 
                System.out.println("Action ID spatial is " + actionSpatialID);
                if(actionSpatialID.contains("living")) 
                    changeScene(livingroom);
                else if(actionSpatialID.contains("kitchen")) {
                    changeScene(kitchen);
                } else if(actionSpatialID.contains("bath")) {
                    changeScene(bathroom);
                } else if(actionSpatialID.contains("bedr")) {
                    changeScene(bedroom);
                }
            }
            
            if(actionSpatial.contains("switch")) {
                // this helps deal with differing linked objects
                String actionSpatialID = gae.getSpatial().getParent().getParent().getName(); 
                if(actionSpatialID.contains("lam")) {
                    if(hallwayLampSO == null) {
                    } else {
                        hallwayLampSO.doAction();
                    }
                } else { // it's just the normal switch. Two switches in the room so need to differentiate. 
                    hallwayCeilingSO.doAction();
                }
            }else if(actionSpatial.contains("tap")) {
                MainGameAppState.getMyPlayer().eat(null);
            }
            
            
        } 
        return false;
    }
    
    /**
     * Sets up all sensory objects in the scene and assigns their properties.
     **/
    public void setupSensoryObjects() {
        // ****** Livingroom sensory objects *******/
        Spatial floorLamp = livingroom.getSpatial("Floor lamp");
        if(floorLamp != null) {
            livingroomFloorLampSO = new LightSensoryObject(floorLamp, "Inside screen2", null, MaterialsManager.getMaterial("transparent"), LightType.STANDARD);
            livingroomFloorLampSO.setUpperLevel(SensoryObject.SensoryState.LOW);
            livingroom.addSensoryObject(livingroomFloorLampSO);
            livingroom.addLampLight(floorLamp);
            livingroomFloorLampSO.setLight(true);
        }
        
        Spatial livingroomCeilingLamp = livingroom.getSpatial("ceilinglamp");
        if(livingroomCeilingLamp != null) {
            livingroomCeilingSO = new LightSensoryObject(livingroomCeilingLamp, LightType.STANDARD);
            livingroomCeilingSO.setUpperLevel(SensoryObject.SensoryState.LOW);
            livingroom.addSensoryObject(livingroomCeilingSO);
            livingroom.addLampLight(livingroomCeilingLamp);
        }
        
        Spatial hoover = livingroom.getSpatial("hoover");
        if(hoover != null) {
            livingroomHooverSO = new SensoryObject(hoover, false, SensoryObject.SensoryType.SOUND_SWITCH, SensoryObject.SensoryState.NONE);
            livingroomHooverSO.setUpperLevel(SensoryObject.SensoryState.HIGH);
            AudioNode hooverNoise = new AudioNode(am, "Sounds/vax.ogg", false); 
            livingroomHooverSO.addSound(hooverNoise, (Node) hoover);
            livingroom.addSensoryObject(livingroomHooverSO);
        }
        
        // ******* Bedroom object setup *******/
        Spatial alarmClock = bedroom.getSpatial("clockface");
        if(alarmClock != null) {
            bedroom_alarmSO = new SensoryObject(alarmClock, false, SensoryObject.SensoryType.SOUND_TEMP, SensoryObject.SensoryState.NONE);
            bedroom_alarmSO.setUpperLevel(SensoryObject.SensoryState.HIGH);
            AudioNode bell = new AudioNode(am, "Sounds/bell.ogg", false);
            bedroom_alarmSO.addSound(bell, (Node)alarmClock);
            bedroom.addSensoryObject(bedroom_alarmSO);
            
            // not the best way to get around the spatial having seperate parts.
            // this assigns it type sensory so it will fire through the action manager.
            // and we can handle it within this class.
            bedroom.assignSpatialActions("clock_body", "sensory");
        }
        
        Spatial bedroomLamp = bedroom.getSpatial("lamp");
        if(bedroomLamp != null) {     
            bedroom_lampSO = new LightSensoryObject(bedroomLamp, LightType.STANDARD);
            bedroom_lampSO.setUpperLevel(SensoryObject.SensoryState.NONE);
            bedroom.addSensoryObject(bedroom_lampSO);
            bedroom.addLampLight(bedroomLamp);
            
        }    
        
        Spatial bedroomCeilingLamp = bedroom.getSpatial("ceilinglamp");
        if(bedroomCeilingLamp != null) {
            bedroom_ceilingSO = new LightSensoryObject(bedroomCeilingLamp, LightType.STANDARD);
            bedroom_ceilingSO.setUpperLevel(SensoryObject.SensoryState.LOW);
            bedroom.addSensoryObject(bedroom_lampSO);
            bedroom.addLampLight(bedroomCeilingLamp);
            
        }
        
        // ******** Bathroom setup ********/
        Spatial toilet = bathroom.getSpatial("toilet");
        if(toilet != null) {
            AudioNode flush = new AudioNode(am, "Sounds/toilet.ogg", false);
            bathroom_toiletSO = new SensoryObject(toilet, false, SensoryObject.SensoryType.SOUND_TEMP, SensoryObject.SensoryState.NONE);
            bathroom_toiletSO.setUpperLevel(SensoryObject.SensoryState.HIGH);
            bathroom_toiletSO.addSound(flush, (Node)toilet);
            bathroom.addSensoryObject(bathroom_toiletSO);
        }
        
        Spatial toothbrush = bathroom.getSpatial("toothbrush");
        if(toothbrush != null) {
            bathroom_toothBrushSO = new SensoryObject(toothbrush, false, SensoryObject.SensoryType.SOUND_TEMP, SensoryObject.SensoryState.NONE);
            bathroom_toothBrushSO.setUpperLevel(SensoryObject.SensoryState.MED);
            bathroom_toothBrushSO.addSound(new AudioNode(am, "Sounds/brushteeth.ogg", false), (Node)toothbrush);
            bathroom.addSensoryObject(bathroom_toothBrushSO);
        }
        
        Spatial bathroomCeilingLamp = bathroom.getSpatial("ceilinglamp");
        if(bathroomCeilingLamp != null) {
            bathroom_ceilingLampSO = new LightSensoryObject(bathroomCeilingLamp, LightType.STANDARD);
            bathroom.addSensoryObject(bathroom_ceilingLampSO);
            bathroom.addLampLight(bathroomCeilingLamp);
        }
        
        // ********* Hallway ********/
        Spatial hallwayLamp = hallway.getSpatial("lamp");
        if(hallwayLamp != null) {
            hallwayLampSO = new LightSensoryObject(hallwayLamp, LightType.STANDARD);
            hallwayLampSO.setUpperLevel(SensoryObject.SensoryState.HIGH);
            hallway.addSensoryObject(hallwayLampSO);
            hallway.addLampLight(hallwayLamp);
            hallwayLampSO.setLight(false);
        }       
        
        Spatial hallwayCeilingLamp = hallway.getSpatial("ceilinglamp");
        if(hallwayCeilingLamp != null) {
            hallwayCeilingSO = new LightSensoryObject(hallwayCeilingLamp, LightType.STANDARD); 
            hallwayCeilingSO.setUpperLevel(SensoryObject.SensoryState.LOW);
            hallway.addSensoryObject(hallwayCeilingSO);
            hallway.addLampLight(hallwayCeilingLamp);
        } 
        
        // ********** Kitchen *********/
        Spatial kitchenCeilingLamp = kitchen.getSpatial("ceilinglamp");
        if(hallwayCeilingLamp != null) {
            kitchenCeilingSO = new LightSensoryObject(kitchenCeilingLamp, LightType.STANDARD); 
            kitchenCeilingSO.setUpperLevel(SensoryObject.SensoryState.NONE);
            kitchen.addSensoryObject(kitchenCeilingSO);
            kitchen.addLampLight(kitchenCeilingLamp);
        } 
        
        Spatial washingMachine = kitchen.getSpatial("Washer");
        if(washingMachine != null) {
            washingMachineSO = new SensoryObject(washingMachine, false, SensoryObject.SensoryType.SOUND_SWITCH, SensoryObject.SensoryState.NONE);
            washingMachineSO.setUpperLevel(SensoryObject.SensoryState.MED);
            AudioNode wash = new AudioNode(am, "Sounds/washingmachine.ogg", false);
            washingMachineSO.addSound(wash, (Node) washingMachine);
            kitchen.addSensoryObject(washingMachineSO);
        }
        
        Spatial fluLight = kitchen.getSpatial("flurescentlight");
        if(fluLight != null) {
            fLightSO = new LightSensoryObject(fluLight, LightType.FLIGHT);
            fLightSO.setUpperLevel(SensoryObject.SensoryState.MED);
            kitchen.addSensoryObject(fLightSO);
            kitchen.addLampLight(fluLight);
        }
        
        
    }
    
    public ArrayList<GameActionEvent> getCustomGameActionEvents() {
        ArrayList<GameActionEvent> gameActions = new ArrayList<GameActionEvent>();
        return gameActions;
    }
    
}

\end{lstlisting}

