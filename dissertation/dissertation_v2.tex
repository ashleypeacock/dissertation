\documentclass[11pt]{report}
\usepackage{fullpage}
\usepackage{graphicx}
\usepackage{float}
\restylefloat{table}
\begin{document}

\title{Increasing perception and understanding of Autism by visual experience}
\author{Ashley Peacock}
\maketitle

\textbf{Abstract}\\
--add this when i'm finished 

\begin{quote}
"if you deal with 'challenging behaviours' in autism, do not focus on the iceberg; do understand the underlying causes of the behaviours and try to develop an approach not based on symptoms but on prevention. Challenging behaviours are caused by problems of communication, social understanding, by different imagination, by sensory problems...Therefore try to understand autism 'from within'. It is easier said than done, because it requires an enormous effort of imagination: we need t learn to put ourselves in the brains of autistic people and then we will understand better through their eyes the obstacles in their attempts to survive among us" - Theo Peeters \cite{olgab}
\end{quote}

\tableofcontents

\chapter{Introduction}

\section{Selecting a project}
The project started with the purpose of creating software to benefit someone with Autism, ADHD or those in contact with these conditions such as family members or carers. Both Autism and ADHD are developmental disorders that start from birth and affect the individual's attention, concentrations and ability to fit into mainstream society.

Owing this was a very broad topic with many possible avenues, multiple proposals were put forward and a selection was made following an online questionnaire, speaking to professionals and parents of children with Autism and ADHD at the ADDISS Conference(2012) and consideration of factors such as the learning curve and plausibility of each project given time constraints.

Project proposals:
\begin{table}[H]
    \begin{tabular}{| p{2cm} | p{5cm} | p{4cm}| p{6cm} |}
    \hline
    Proposal name & Description &  For & Against \\
    \hline
    \hline
    Online diary & Online system to improve communication between carers, parents, social workers, schools. Parties could post questions and ask for suggestions when dealing with certain behaviours as well as document the child's day allowing easier identification of patterns of behaviour or problems & 
   Seamless communication between doctors, teachers and carers which is problematic and information can be missed.
   & \begin{minipage}{5cm}
    \vskip 4pt
    \begin{enumerate}
   \item Good in theory but may not be practical due to data protection.
   \item Relies too heavily on parents/carers being able to read emails or notifications.
   \item May be difficult for some schools to gain access to wifi.
   \end{enumerate}
   \vskip 4pt
 \end{minipage}                        \\
    \hline
    Social simulator & Simulated social scenarios for autistic users to trial various social situations and see possible outcomes whilst being given potential tips and strategies & & \begin{minipage}{5cm}
    \vskip 4pt
    \begin{enumerate}
   \item Big project given time-frame
   \item Other companies working on similar concepts and much research has been done on this topic already.
   \end{enumerate}
   \vskip 4pt
 \end{minipage}     \\
    \hline
    Dynamic scheduler and planner app & A planner that would re-schedule tasks when not completed and present basic to-do lists with tasks broken down into manageable chunks &
    No planners available that specifically target planning/executive functioning difficulties within ADHD/Autism.
  &
Least unique proposal, many other planners available.
  \\
    \hline
    Environment app & Phone app aimed to encourage children to engage with the environment around them with simple questions and pictures: "Can you see a blue car?".  & 
    Least amount of implementation work, could be simple but effective.
 & \begin{minipage}{5cm}
    \vskip 4pt
    \begin{enumerate}
   \item Hard to back up with literature 
   \item Difficult concept to understand
   \end{enumerate}
   \vskip 4pt
 \end{minipage}    \\
    \hline
   Autism simulator & A 3D virtual environment where the user plays as a child with autism and can thus experience some of the obstacles faced through a visual/game environment & 
   \begin{minipage}{4cm}
    \vskip 4pt
    \begin{enumerate}
   \item Most unique and popular idea
   \item Misunderstanding from public is a big problem
   \item Could be extremely helpful in aspects such as teacher's training which is expensive.
   \end{enumerate}
   \vskip 4pt
 \end{minipage}   &
 \begin{minipage}{5cm}
    \vskip 4pt
    \begin{enumerate}
   \item Big project given the time frame, no previous simulators at the time of selection that could be drawn from.
   \item  No evidence or backing from literature available
   \end{enumerate}
   \vskip 4pt
 \end{minipage}    \\
    \hline
    \end{tabular}
\end{table}

 The planner was eliminated on the basis of being the least unique concept with many currently available. Descriptions of four of the projects were put on a website and people were asked to fill out a questionnaire of their preferences. It was completed anonymously by five people in total and included people with ASD/ADHD, professionals, carers and parents. Questions put forward were:

\begin{enumerate}
\item Please give some information about yourself, for example if you have ASD/ADHD or are a professional/carer.
\item Please select and rank three proposals you feel are the best
\item Please explain reasons for selection
\end{enumerate}

\begin{table}[H]
    \begin{tabular}{| p{2cm} | p{3cm} | p{3cm}| p{3cm} | p{4cm} |}
    \hline
    No. & Rank 1 & Rank 2 & Rank 3 & Comments on candidate \\ \hline
    1 & Autism Simulator & Social Simulator & Diary & PHD student and project supervisor for informatics UG4 projects at Edinburgh University\\ \hline
    2 & Autism Simulator & Diary & Social simulator & Parent of two children with ASD/ADHD. Works professionally with young people with disabilities and their carers\\ \hline
    3 & Social Simulator & Autism Simulator & Diary & Parent of two children with ASD\\ \hline
    4 & Autism Simulator & Social Simulator & Environment app & Adult with Autism. \\ \hline
    5 & Social simulator & Autism simulator & Diary & Not specified \\ \hline
    \end{tabular}
\end{table}

Participants 1 and 2 gave individual written feedback on each project as well as completing the survey. In addition, one person chose to give feedback on the individual projects rather than filling out the questionnaire. This person is a professional and counsellor to neurodiverse adults and has setup support groups and workshops for many years. 

Comments on the individual projects can be summarised below:

\begin{table}[H]
    \begin{tabular}{| p{3cm} | p{12cm} |}
    \hline
    Project name & Comments \\ \hline
    Autism Simulator & Most highly thought of concept. Worries about the concept being far too big. A book called "skallagrigg" which a person with cerebral palsy creates a similar game and was the topic of an AS reading group. People in the group said that they would love for such a thing to exist\\ \hline
    Environment app & Generally quite difficult to back up with literature. Concept was generally difficult to understand and not well explained on the website. However, commented that as children with autism tend to love technology/ipads it could provide a motivator to access the world and help deal with overwhelming stimuli\\ \hline
    Social simulator & Social situations are too unpredictable and hence social simulations tend to be catered for the individuals however, giving strategies and suggestions could work quite well. There's also lots of others working on these concepts and it already had a large base of literature demonstrating the challenge to the task. \\ \hline
    Diary & Data protection could be an issue. Limited use of Wifi and computers in school could make it inaccessible. In a play-scheme context it is a good idea in theory, but again getting use of a computer would be difficult. A phone/text system might work better. People also tend to include opinions and perspectives of situations and this may present additional problems. \\ \hline
    \end{tabular}
\end{table}

From the results of the questionnaire it became obvious which of the three concepts people felt were the most beneficial although the Environment app's score may have suffered due to not being particularly well explained. One of the key goals for this project is to create and artefact that can be used within the community and as such the "Diary" was eliminated on the basis of data protection and confidentiality problems. This left two projects the autism and social simulator. The autism simulator was selected as from all sources, responses were the most positive with the only concern being its potential size and lack of restriction, which could be eliminated by conveying a subset of autistic difficulties and if a game engine was selected for development rather than creating the graphics engine from scratch it should alleviate any concerns of the time restraints.

\chapter{Literature review}

\section{What is Autism?}

\subsection{DSM-5}

\subsection{Public perception}
Figures drawn from the 2011 census estimate that 1.1\% of the population have Autism\cite{nas} and as this figure has greatly risen\cite{increasingprevalence} so too has the need for greater understanding\cite{autism_awareness}. Consequently, millions of pounds have been spent on campaigns across the globs such as World Autism Awareness Day and Light it up blue organised by Autism speaks(2013)\cite{autism_awareness}.

A survey published by the National Autistic Society revealed that 92\% of respondents had heard of Autism but only 48\% had heard of Aspergers syndrome which has less obvious difficulties. Research has indicated that general members of the public are aware of communication and social issues that come with autism\cite{autismmisconception}, but little are aware of sensory difficulties\cite{autism_awareness}. In \cite{autism_awareness}, of 1204 people surveyed, 293 were aware of communication difficulties, 131 social but only 12 were aware of sensory difficulties. These is alarming owing "Many people with Asperger syndrome/High functioning autism define their sensory processing problems as more disabling than the deficits in communication/social behaviour\cite{olgab}.

\begin{quote}
If I could make one change... every person who comes into contact with my daughter would have some form of training in autism.\cite{nasschool}
\end{quote}


\section{Sensory processing}
Sensory processing differences in autism are highly reported, 81\% of respondents reported differences in visual perception, 87\% in hearing, 77\% in tactile perception, 30\% in taste and 56\% in smell \cite{sensory_leisure}. Senses play a vital role in how we model and perceive the world around us so if one senses the world in a differently, their view and resulting behaviours will also be different. 

Senses in autism can be hyper(more sensitive), hypo(less sensitive), agnostic or fluctuate between hyper and hypo\cite{bayes}. Fluctuations can be described as a "FM radio that is not exactly tuned on the station when you are driving down the freeway. Sometimes the world comes in clearly and at other times it does not" \cite{olgab}. As with all areas of autism, sensory atypicalities differ and are unique to each individual, however, fluctuations create even greater challenges for carers and for a person with autism to identify or predict troubling sources before they occur. 

When a sensory channel is in a state of agnosia, although able to see, one may not be able to assign it to any meaning. The result is one can become 'mind-blind', or 'mind-deaf' and as a result, a person with autism can act as if they are genuinely deaf. Below are examples of the effects someone with autism may experience depending on the state of their sensory channel:

\begin{table}
    \begin{tabular}{| l | p{5cm} | p{5cm} |}
    \hline
    Sense channel & Hyper                                                                                                                      & Hypo                                                                   \\
    \hline
    \hline
    Vision        & Vision may be magnified                                                                                                    & Attracted to light or fascinated with bright colors                    \\
    \hline
    Auditory      & Sounds are amplified. Temple Grandin a write with autism described her ears as like 'microphones'                          & Is attracted to sounds/noises                                          \\
    \hline
    Tactile       & Clothes may hurt. One person with autism described clothe labels as feeling like 'barbed wire'. May not like being hugged. & Enjoys being hugged or seeks pressure by crawling under heavy objects. \\
    \hline
    Taste/Smells & Smells or texture of foods may be intolerable. & Mouths and licks objects \\
    \hline
    Vestibular & Difficulty with walking or crawling on uneven or unstable surfaces. & Spins, runs round and round, rocks back and forth \\
    \hline
    \end{tabular}
\end{table}

Sensory processing patterns can be categorised into four-types\cite{sensory_leisure}:

Correlation between sensory difficulties and difficult temperament characteristics such as activity level, adaptability to changing context, quality of mood, threshold of responsiveness, intensity of reaction and persistence\cite{temperament}. 

\subsection{Resulting behaviours}

\subsubsection{Meltdowns}

\subsubsection{Sensory overloads}

\subsubsection{Unusual fears}
It was found that 40\% of children with autism had unusual fears in comparison to 0-5\% of typical children and the vast majority of these fears consisted of mechanical objects. Children with autism have higher levels of anxiety than typical children\cite{fears} and increased anxiety from being faced with more fears on a day to day basis will only increase stress. "The fear that it might happen can be as bad as it actually happening" and even if the cause is identified and removed, for example not flushing the toilet whilst the child is in the bathroom, it can take considerable time for the worry to go away.

Perceived unusual fears could include leaving the house because it's cloudy and a fear of rain, not taking a shower because of the noise from the drain, not going to school due to being afraid the fire alarm will sound. The top five reported unusual fears were toilets, elevators, vacuum cleaners, thunderstorms, tornadoes. The cause of many of these unusual fears in children with autism are thought to be related to sensory perception differences\cite{fears}.

\section{Coping mechanisms}
- restricted, repetitive behaviours
- routines
- spinning

\section{Theoretical models}
\subsection{Information processing}
\subsection{Perceptual models}

\section{Impact of Autism}
One person with Aspergers syndrome(a form of high-functioning autism) described the experience as like "living in a bubble or living on the other side of a plate glass window to everybody else. It is like you are just a spectator in this thing"\cite{aspieway}. From interviews conducted by Sara Ryan and Ulla Raisanen(2008) three themes emerged: not belonging, trying to fit in and the need for safe spaces. Inspite of this, interviews showed their desire was not to rid themselves of Aspergers but to simply fit into main-stream society. Interviewees were extremely aware of their differences but inspite of desperately trying to learn the rules and social norms it was often felt their efforts were not reciprocated by neurotypical people.

Of course, one solution to aid those on the Autistic spectrum to fit into main-stream society would be increase public awareness, acceptance and understanding. However, for people with Autism, explaining emotions and feelings with words was described as painful and thus giving others this understanding is difficult \cite{aspieway}. 

\subsection{Impact of Autism in Schools}
It is estimated that only 22\% of teachers have been trained specifically in autism and the majority of training given is typically one to four hours. 54\% of all teachers in England do not feel they have had adequate training to teach children with autism.\cite{statsandfacts} 30\% of parents of children with autism in mainstream education are satisfied with the level of understanding of autism across the school\cite{nasschool}. 23\% of parents are dissatisfied with SENCO's level of understanding of autism. 

Figures obtained show that approximately 40\% of children with autism have been bullied at school. 1 in 5 children with autism have been excluded from school \cite{nasschool} and only 24.4\% of pupils with autism achieved 5A*-C GCSEs in 2010/2011 in comparison to 58.2\% of the overall population\cite{statsandfacts}, a surprising figure owing people with autism are deemed to have above average intelligence, indicating difficulties at school may be a reason for not for-filling potential. 

\begin{quote}
If I could make one change...I would ensure compulsory, thorough training about autism and how it affects learning is given to all school staff. \cite{nasschool}
\end{quote}

-- need to write here about the more detail of what it's like being affected in school and how schools struggle with challenging behaviours. 


\begin{quote}
Danny would not have been excluded if the school had understood the difference between 'normal' behaviour and Aspergers syndrome. They inflamed situations because they didn't understand that my son finds physical contact, or being touched by teachers, really difficult \cite{nasschool}
\end{quote}

\subsection{Impact of Autism on Home}
Parents of children with Autism describe outings as being extremely challenging, not only because of the unpredictable nature of meltdowns, but because of unpredictable public reactions\cite{meltdowns_goingout} commented as "the hardest thing to deal with"\cite{meltdowns_goingout}.

Often, parents would want to react simultaneously in multiple ways, anger, frustration, wanting to explain but instead shutting down themselves and simply ignoring members of the public and trying to get away from the situation\cite{meltdowns_goingout}. Competent parents are often seen to be incompetent when managing meltdowns which on the surface can appear like temper-tantrums and parents are often left with feelings embarrassment\cite{meltdowns_goingout}. Parents expressed that if they explained to members of the public, the response was more positive but it is extremely tiring and time consuming to do this\cite{meltdowns_goingout}. Some have responded giving out business cards issued by the National Autistic Society which contains some information and websites about autism, but sometimes if the attention is too great there is simply not enough. Sometimes members of the public could also be left feeling embarrassed and ashamed of themselves after realising the child had autism\cite{meltdowns_goingout}.  

To support children with autism when going out and about, parents found that giving notice and preparation to the child worked quite well, but when stressed of they forgot, it could lead to a meltdown and even more stress\cite{meltdowns_goingout}. Meltdowns can just told hold of the child with no obvious cause although through time and practice they can become easier to predict. Many parents link their children's disruptive behaviours to sensory difficulties, and in the unpredictable outside world full of bright lights, unusual and loud noises, even a simple tasks such taking the child to the toilet can become a challenge if, for example the hand-dryer is unexpectedly switched on\cite{meltdowns_goingout}. Common family outings such as going to the pictures are impossible due to sounds and fears of darkness and this in turn can have an impact on siblings.  

Lack of understanding applies not only to the outside world, but even with family members\cite{meltdowns_goingout}. Parents may be unable to go to special occasions such as Birthdays due to fear of meltdowns and disapproval. If no-one can be found to baby-sit, it means they too cannot attend and can be left feeling further isolated.

\chapter{Conclusions}

\section{Future work}

\begin{enumerate}
\item Occulus rift
\item Mind headset
\item Supporting others using system to build their own experiences
\item Research tool.
\end{enumerate}

\begin{thebibliography}{9}

\bibitem{increasingprevalence}
Johnny L. Matson, Alison M. Kozlowski
The increasing prevalence of autism spectrum disorders. Research in Autism Spectrum Disorders(2011)

\bibitem{nas}
National autistic society. www.autism.org.uk

\bibitem{statsandfacts}
Ambitious about autism. www.ambitiousaboutautism

\bibitem{nasschool}
Make school make sense. Autism and education: the reality for families today. National Autistic Society, 

\bibitem{autismmisconception}
Autism misconceptions. NHS

\bibitem{olgab}
Sensory Perceptual Issues in Autism and Aspergers syndrome(2003). Bogdeshina, O

\bibitem{fears}
Unusual fears in children with autism(2012). Susan Dickerson Mayes, Susan L. Calhound, Richa Aggarwal, Courtner Baker, Santosh Mathapati, Sarah Molitoris, Rebeccas D. Mayes.

\bibitem{temperament}
Sensory correlates of difficult temperament characteristics in preschool children with autism(2012). I-Ching Chaung, Mei-Hui Tseng, Lu Lu, Jeng-Yi Shieh

\bibitem{bayes}
When the world becomes 'too real': a Bayesian explanation of autistic perception(2012). Elizabeth Pellicano and David Blurr.

\bibitem{williams1996}
Autism: An inside-out approach(1996). Williams, D

\bibitem{williams1994}
Somebody Somewhere: Breaking Free from the World of Autism(1994). Williams, D.

\bibitem{williams1992}
Nobody no-where(1992). Williams, D

\bibitem{sensory_leisure}
Sensory processing abilities and their relation to participation in leisure activities among children with high-functioning autism spectrum disorder. Hochhauser, M. Engel-Yeger, B.

\bibitem{sensory_toddlers}
Parent Reports of Sensory Symptoms in Toddlers with Autism and Those with Other Developmental Disorders. Sally J. Rogers, Susan Hepburn, Elizabeth Wehner

\bibitem{sensory_children}
Describing the sensory abnormalities of children and adults with autism(2007). Susan R. Leekam. Carmen Nieto. Sarah J. Libby. Lorna wing. Judith gould. 

\bibitem{sensory_perceptual}
Sensory integration and the perceptual experience of persons with autism(2006). Grace Iarocci. John McDonald.

\bibitem{rss_cognitive}
Restricted and Repetitive Behaviours, Sensory Processing and Cognitive Style in Children with Autism Spectrum Disorders(2009). Yu-Han Chen, Jacqui Rodgers, Helen McConachie.

\bibitem{rrsyouth}
Restricted and repetitive behaviours and psychiatric symptoms in youth with autism spectrum disorders(2013) Elizabeth A. Stratic, Luc Lecavlier.

\bibitem{rrs_sensory}
Is there a relationship between restricted, repetitive, sterotyped behaviours and interests and abnormal sensory response in children with autism spectrum disorders?(2008). Robin L. Gabriels, John A. Agnew, Lucy Jane Miller, Jane Gralla, Juliet P. Dinkins, Elizabeth Hooks.

\bibitem{aspieway}
"It’s like you are just a spectator in this thing": Experiencing social life the 'aspie' way(2008). Sara Ryan, Ulla Raisanen

\bibitem{dd}
Disability awareness: Beyond the day: http://www.serviceandinclusion.org/index.php?page=simulations. Danielle Dreilinger

\bibitem{autism_awareness}
Awareness and knowledge of autism and autism interventions: A general population survey. Karola Dillenburger(2013). Julie Ann Jordan, Lyn McKerr, Paula Devine, Mickey Keenan

\bibitem{meltdowns_goingout}
'Meltdowns', surveillance and managing emotions: going out with children with autism(2010). Sara Ryan.

\bibitem{sensory_overview}
Sensory issues in Autism(2007). East sussex county council. 

\end{thebibliography}

\end{document}