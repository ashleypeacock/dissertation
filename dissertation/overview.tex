\chapter{Overview}

The projected started by firstly creating and considering various project proposals that may benefit people with ADHD or Autism. Following this, research and feedback aided the selection of the project an "Autism Simulator" whereby the user plays as a child with Autism and gets to experience some of the difficulties, specifically sensory related difficulties through their eyes. Consultation with the Learning and Adaptive Environment Research(LAER lab) in addition to interviews with people on the Autistic spectrum, professionals and school teachers further solidified this selection and gave indication of the challenges faced as to inform design choices and goal selection. A prototype of the simulator was then created using a Game Engine for speedy development and the project was evaluated by the LAER group with an additional evaluation in the form of an on-line survey where participants were able to view a video demo of the simulator and give feedback. The first playable version was subsequently created after an extensive re-write and addition. A final formative evaluation was conducted with various students to aid game-play decisions and improvements before a summative evaluation was completed by various members of the university involved in education. 

\section{Selecting a project}
The project started with the purpose of creating software to benefit someone with Autism, ADHD or those in contact with these conditions such as family members or carers. Both Autism and ADHD are developmental disorders that start from birth and affect the individual's attention, concentrations and ability to fit into mainstream society.

Owing this was a very broad topic with many possible avenues, multiple proposals were put forward and a selection was made following an online questionnaire, speaking to professionals and parents of children with Autism and ADHD at the ADDISS Conference(2012) and consideration of factors such as the learning curve and plausibility of each project given time constraints.

Project proposals:
\begin{table}[H]
    \begin{tabular}{| p{2cm} | p{5cm} | p{4cm}| p{6cm} |}
    \hline
    Proposal name & Description &  For & Against \\
    \hline
    \hline
    Online diary & Online system to improve communication between carers, parents, social workers, schools. Parties could post questions and ask for suggestions when dealing with certain behaviours as well as document the child's day allowing easier identification of patterns of behaviour or problems & 
   Seamless communication between doctors, teachers and carers which is problematic and information can be missed.
   & \begin{minipage}{5cm}
    \vskip 4pt
    \begin{enumerate}
   \item Good in theory but may not be practical due to data protection.
   \item Relies too heavily on parents/carers being able to read emails or notifications.
   \item May be difficult for some schools to gain access to wifi.
   \end{enumerate}
   \vskip 4pt
 \end{minipage}                        \\
    \hline
    Social simulator & Simulated social scenarios for autistic users to trial various social situations and see possible outcomes whilst being given potential tips and strategies & & \begin{minipage}{5cm}
    \vskip 4pt
    \begin{enumerate}
   \item Big project given time-frame
   \item Other companies working on similar concepts and much research has been done on this topic already.
   \end{enumerate}
   \vskip 4pt
 \end{minipage}     \\
    \hline
    Dynamic scheduler and planner app & A planner that would re-schedule tasks when not completed and present basic to-do lists with tasks broken down into manageable chunks &
    No planners available that specifically target planning/executive functioning difficulties within ADHD/Autism.
  &
Least unique proposal, many other planners available.
  \\
    \hline
    Environment app & Phone app aimed to encourage children to engage with the environment around them with simple questions and pictures: "Can you see a blue car?".  & 
    Least amount of implementation work, could be simple but effective.
 & \begin{minipage}{5cm}
    \vskip 4pt
    \begin{enumerate}
   \item Hard to back up with literature 
   \item Difficult concept to understand
   \end{enumerate}
   \vskip 4pt
 \end{minipage}    \\
    \hline
    \end{tabular}
\end{table}

\begin{table}[H]
    \begin{tabular}{| p{2cm} | p{5cm} | p{4cm}| p{6cm} |}
    \hline
    Proposal name & Description &  For & Against \\
    \hline
    \hline
   Autism simulator & A 3D virtual environment where the user plays as a child with autism and can thus experience some of the obstacles faced through a visual/game environment & 
   \begin{minipage}{4cm}
    \vskip 4pt
    \begin{enumerate}
   \item Most unique and popular idea
   \item Misunderstanding from public is a big problem
   \item Could be extremely helpful in aspects such as teacher's training which is expensive.
   \end{enumerate}
   \vskip 4pt
 \end{minipage}   &
 \begin{minipage}{5cm}
    \vskip 4pt
    \begin{enumerate}
   \item Big project given the time frame, no previous simulators at the time of selection that could be drawn from.
   \item  No evidence or backing from literature available
   \end{enumerate}
   \vskip 4pt
 \end{minipage}    \\
    \hline
    \end{tabular}
\end{table}

The planner was eliminated on the basis of being the least unique concept with many currently available. Descriptions of the four remaining projects were put on a website and people were asked to complete a questionnaire with their preferences. Participants were asked to rank 1-3 which proposals they felt would be the most beneficial to the community as well as answering the following quantitative questions: 

\begin{enumerate}
\item Please give some information about yourself, for example if you have ASD/ADHD or are a professional/carer.
\item Please select and rank three proposals you feel are the best
\item Please explain reasons for selection
\end{enumerate}

\subsection{Results}
It was completed anonymously by five people in total and included people with ASD/ADHD, professionals, carers and parents:

\begin{table}[H]
    \begin{tabular}{| p{2cm} | p{3cm} | p{3cm}| p{3cm} | p{4cm} |}
    \hline
    No. & Rank 1 & Rank 2 & Rank 3 & Comments on candidate \\ \hline
    1 & Autism Simulator & Social Simulator & Diary & PHD student and project supervisor for informatics UG4 projects at Edinburgh University\\ \hline
    2 & Autism Simulator & Diary & Social simulator & Parent of two children with ASD/ADHD. Works professionally with young people with disabilities and their carers\\ \hline
    3 & Social Simulator & Autism Simulator & Diary & Parent of two children with ASD\\ \hline
    4 & Autism Simulator & Social Simulator & Environment app & Adult with Autism. \\ \hline
    5 & Social simulator & Autism simulator & Diary & Not specified \\ \hline
    \end{tabular}
\end{table}

Participants 1 and 2 gave individual written feedback on each project as well as completing the survey. In addition, one person chose to give feedback on the individual projects rather than filling out the questionnaire. This person is a professional and counsellor to neurodiverse adults and has setup support groups and workshops for many years. 

Comments on the individual projects can be summarised below:

\begin{table}[H]
    \begin{tabular}{| p{3cm} | p{12cm} |}
    \hline
    Project name & Comments \\ \hline
    Autism Simulator & Most highly thought of concept. Worries about the concept being far too big. A book called "skallagrigg" which a person with cerebral palsy creates a similar game and was the topic of an AS reading group. People in the group said that they would love for such a thing to exist\\ \hline
    Environment app & Generally quite difficult to back up with literature. Concept was generally difficult to understand and not well explained on the website. However, commented that as children with autism tend to love technology/ipads it could provide a motivator to access the world and help deal with overwhelming stimuli\\ \hline
    Social simulator & Social situations are too unpredictable and hence social simulations tend to be catered for the individuals however, giving strategies and suggestions could work quite well. There's also lots of others working on these concepts and it already had a large base of literature demonstrating the challenge to the task. \\ \hline
    Diary & Data protection could be an issue. Limited use of Wifi and computers in school could make it inaccessible. In a play-scheme context it is a good idea in theory, but again getting use of a computer would be difficult. A phone/text system might work better. People also tend to include opinions and perspectives of situations and this may present additional problems. \\ \hline
    \end{tabular}
\end{table}

\subsection{Conclusions}
From the results of the questionnaire it became obvious which of the three concepts people felt were the most beneficial although the Environment app's score may have suffered due to not being particularly well explained. One of the key goals for this project is to create and artefact that can be used within the community and as such the "Diary" was eliminated on the basis of data protection and confidentiality problems. This left two projects the autism and social simulator. The autism simulator was selected as from all sources, responses were the most positive with the only concern being its potential size and lack of restriction, which could be eliminated by conveying a subset of autistic difficulties and if a game engine was selected for development rather than creating the graphics engine from scratch it should alleviate any concerns of the time restraints. The first limitation from the outset was to restrict the simulator to a home environment and this selection will be discussed in slightly more detail. 