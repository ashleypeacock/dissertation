\chapter{Discussion and final conclusions}

The overall project started by selecting the project; through speaking to professionals at the ADDISS conference and after the initial selection through interviews this was reduced these to a set of goals.  
From this a prototype version was created and formally evaluated by experts in the field as well as parents of children autism and people with autism themselves. Following feedback and identified changes stories were created with consultation from someone with autism whom also recorded many of the sounds that were used.
The first version was subsequently created although due to performance issues the redesign was much greater than anticipated; not only was there substantial changes to the code the entire house required remodelling on an even greater level than before. 
Formative evaluations were then conducted to target potential game play problems and to create a project that could be "picked up and played" with minimal materials and instructions required such that people were able to learn mostly from experience, trial and error and trying out different strategies. This was an iterative process to identify which problems were frequently appearing and through the formative process it is felt that improvements in participants were seen in terms of their ability to primarily focus on the simulator and learning goals rather than game-play elements. Following this a summative evaluation was conducted with teachers and people from an education background to see if there was an overall improvement in understanding of autism. However owing many came from autism backgrounds not much of an improvement was seen although understanding definitely was not made worse.

\section{Review of aims}

Does the project reach its goals?
\begin{enumerate}
\item Can a simulator be developed that is successful in allowing users to play as an autistic child and thereby experience some of the difficulties/obstacles faced?
\item How was a particular focus on sensory difficulties built in to this?
\item Does the game allow the user to obtain some useful information on how possible household items/situations can trigger sensory problems, as well as how a child with autism might cope with these?
\item Does the simulator increase understanding and awareness of autism and the difficulties faced?
\end{enumerate}

\subsection*{Aim 1}
A simulator was created with the goals of allowing users to play as a child with autism and came with experiencing some of the specifically, sensory difficulties faced. Apart from the height of the character in relation to the rest of the game however and the special interest of a dinosaur not much particularly distinguishes the simulation from an adult with autism. Most information in sensory difficulties was obtained from adults whom are better adapted to dealing with these. Research would be required to give a better all around representation of how a child might deal with these, possibly including perceived sensory alleviating features such as repetitive movements and behaviours.  

\subsection*{Aim 2}
Sensory difficulties were given the most attention both during development and throughout game play. It is the only aspect which the user really experiences; other aspects of autism are explained by using description boxes which should give additional information to those whom want it but are not the most salient aspect. 

\subsection*{Aim 3}
The user can obtain useful information on how possible household objects can trigger sensory problems by experiencing them or acquiring information from description boxes:

Sensory overloads are conveyed with visual distortions, lights get brighter and more obtrusive and for sounds direct recordings of sounds are used with a high volumne although the user can of course turn these down if they are too problematic. Some senses were implemented by using descriptions and thoughts such as smell or food textures so when the user tries to eat cheese or chicken thoughts appear of their dislike. Tactile representations were added such as the toothbrush and action of getting dressed; however it is felt more needs to be done to represent this and the "getDressed" needs better explaining, a better hint or a more obvious animation such as the wardrobe opening.

The simulator conveys the effects of sensory overloads by an implementation of meltdowns in two ways; freeze and fight. Freeze was a suggestion from someone with autism and was later chosen not to be used in the public release as it was uncertain if this aspect had been accurately translated. A person with autism suggested this could happen during sensory overloads and because they could not move it would lead to a meltdown rather than freezing being a direct response to a meltdown. 

Finally in terms of learning about how a child may deal with these problems, the morning routine gives some indication and explanation that routines are an important feature of autism to alleviate the problem but it is felt more can be done. Much of the simulator is left for the user to think of their own conclusions and strategies by discovery; there is no explicit hint given that users could walk around looking at the ground to alleviate light sensory problems.


\subsection*{Aim 4}
Although conclusions from the summative evaluation suffer from the ceiling effect there is evidence of at least some of an increase in understanding of autism. The aim is unfortunately only partially achieved and it is felt that more research is required, ideally not with people whom come from having had experience with autism. By making a website available and having the system readily accessible this should aid increasing awareness of autism and as later expressed in \ref{sec:publicresponse}, the website received a large amount of viewers in the space of a few days. However, the aim of putting this completely online so users would not have to download it was not reached as it was not possible to protect the source code due to problems with obfuscation. However, the project could possibly be made open source at a later date.


\section{Limitations}

\subsection{Evaluation and data collection}
Multiple evaluations used which was vital. each gave differing information, but even still, more would have been better. Only once the project was complete did I start to acquire new information on sensory overloads and differences not previously read(explained in more detail below)
There were four evaluations in total however more need to done. Initial plans to present this to a group of parents were scarpered due unanticipated overhaul.
Better project design and identification of risks may have been a good way to have gotten around the need to backtrack where consequently not much of the prototype was evident in the final version created apart from overall the architecture. However, being new to both blender and JMonkey made it very difficult to be able to do this. The same can be said about many of the initial design ideas, it was difficult to gauge the time required to complete having not had prior experience with the tools used. Character modelling and animations is an extremely time consuming and model for this had to be taken from blendswap, thus limiting potential animations which could have been useful in the project. However the project was only created up until the afternoon routine and so this didn't hinder too much.

More interviews needed to be completed, better recorded and better planned. Having never conducted interviews before it was very difficult to know what information would be required but as time and experience grew gaps in information missing from interviews became evident. Instead of asking about an overall picture of sensory difficulties, asking specific information such as how lights or sound affects an individual would have been more useful. However, an overall picture of sensory difficulties was what was needed at the time and specific information needs came later thus more interviews with a variety of individuals at a later stage would have been useful although through the initial formative evaluation it was felt sensory effects used were a good representation, at least in terms of explaining to users the negative effects. 

Inspite of this initial interviews were fantastic and a great asset to the project, hearing people with autism speaking of their sensory issues and the genuine feeling of the traumatisation that they can feel because of them which to those of us whom do not have this experience it is extremely difficult to understand. 

However, frequently consultation with people with autism occurred although not always unfortunately recorded. Asking on experiences of people with with autism in relation to sensory overloads or meltdowns was extremely useful and as seen in the lit review the internet can be a fantastic place for those with autism to freely speak; something that may have been made difficult with interviews.

It was found later that by having a visual representation, people with autism were much more able to discuss and give additional information; information that had not previously been obtained from literature or from interviews; 

\begin{quote}
Well done with the stuff on the lights although personally i get more anxious in the dark then the light because i feel like i'm being hunted also what is that banging? You should also implement that when you turn a machine of like the hover you should get told off by your parents you could also add sights like a bird out the window to distract you. Your explanations are very good, the bar goes and replenishes perfectly. - young person with Autism
\end{quote}

The reason could be because those with autism require specific information so with a visual experience you can ask specific information. A visual game gave a basis and a platform to prompt discussion and explain differences or similarities in order to discuss and explain. 


\subsection{Simulator}
From feedback the system is still too difficult and this can impact and take away from the learning experience. It is felt the difficulty requires direct attention and trials will multiple individuals. Participants in the formative evaluation were all university students whom had experience with games and thus found it much easier and it was this basis the difficulty was unfortunately set. With more scenarios and game-play this can probably be reduced and potentially increased as the story moves onwards.
 
\subsubsection{Sensory system}
Depicting sensory overloads and designing and algorithm to do this was difficult. There is a vast amount of differing literature on the topic, vast amount of different experiences and as people with autism have expressed, sensory differences can be extremely difficult to express. Interviews allowed some insight into this, videos and feedback helped with more, and later feedback aided tweaks. But it is still felt more research and feedback is required. There were features from the prototype which are preferred to the first version design such as a better system delay in sensory overload occurring and slow building of a sensory overload to give the user warning to move away. However, feedback in relation to the sound sensory effects suggested depiction of sounds was extremely accurate. 


\subsection{Use of tools}

\subsubsection{JMonkey}
Offered a lot for importing models but a substantial amount of time in the first year spent on simply playing around with models just to get them imported. In late 2013 the importer was greatly improved and this coupled with reducing sizes of scenes by compartmentalising the house massively improved time able to spend directly on development itself.
However inspite of using a game engine, a lot of code still had to be written(nearly 5 thousand lines). Being able to focus on the high level aspects was the main reasons for choosing a game engine but focus on higher-level aspects was not possible until the last month of development when a solid basis had been written and feedback was directed to improvements rather than solving problems.
Unity would definitely have obtained similar results far quicker and with much, much less code however it is still felt that in the long run JMonkey has a lot more to offer with an extremely active community in developers themselves frequently respond with help.


\subsubsection{Blender}
Blender was a fantastic program to use and is widely used by professionals. It did have an exceptionally steep learning curve and time is regretted to have been spent on sketchup in the prototype state when it could have been spent learning to use this and made modelling quicker in the second year of this project. Far too much time was spent on the prototype finding models which could be used and then spending a large amount of time trying to clean them up so they could be imported into JMonkey and it is felt this time would also have been much better spent on learning blender. The new living room took three days to model however when it came to the kitchen which was much more detailed this was done in only 3 hours due to increased experience.

\section{Future directions}
In addition to suggested future improvement in the summative and formative: additional work is required for sound sensory. Pre-recorded sounds not that were not fully implemented in the first version need to be implemented and some additional ones need to be recorded such as sounds outside such as cars or dogs when walking around the house. Particle emitters were implemented from the prototype although not directly used in the final version. This was because feedback on the use of these was desired and owing people with autism expressed already "the sound is very accurate" it was decided not to include these and was left for a future question and improvement. 

Improving animations; some users expressed they expected the door of the wardrobe to open although there was no need to implement this and a representation of getting dressed was indicated by the images more realistic animations and interactions at a later date are advised to create a further immersive environment. For example making the dinosaur move instead of simply rotation. 

Further interactions of items without explicitly having to interact with them; i.e looking at the windmill and if the user tries to look away the camera is brought back to it. Thus directing the in-game players attention will be difficult to completely control.  

Breaking down explore mode and having both an explore mode and a tutorial mode. Due to time explore mode became a bit more like a tutorial mode although the two were never fully separated and combined. 

Interface needs further improving as thoughts are still not obvious. The tool tip which has been centred under the cross hair is a large improvement however it still needs to be further obvious. 

\subsection{Scenario and scene customisability}
More scenarios and stories need to be created and I feel that having now completed the original process and story I am better equipped to know which questions specific information needs to be requested in interviews or generally when designing the next phases. 

In game characters could have obscure body language or skills could be acquired as missions go on, i.e acquiring "stimming" as an action that can be used as long as the parent is not in the room(otherwise they are asked to stop and this results in another drop in contentment instead of increasing it).

What has been additionally created is a platform for people to be able to create their own scenarios and situations however further compartmentalisation of missions from the rest of the system and better documentation is needed. Ideally it would be good to test this with a few computer scientists before releasing it.

\subsection{Designing for autism}
There needs to be options to turn on and off sensory overload effects and and aspects such as the fluorescent lights which could cause major problems for some users and  potentially hinder feedback from those with autism whom struggled to play due to causing them sensory overloads although none specified they could not play it at all - just with a bit of difficulty. 

There also needs to be more thought in terms of designing the system for people with autism so they can give feedback. The website in some places was found to be confusing with specific information not being given. Such lead to fantastic feedback although individuals reported being confused of content, layout with a more specific explanation of the projects aims and goals:

\begin{quote}
I learned a bit about what it might feel like to have sensory issues but not sure if I learned anything about autism. I would imagine that people who do not have autism but have sensory issues will suffer similarly. You are focusing on symptoms that some people with autism have but others do not. ~ Adult with aspergers syndrome
\end{quote}

In addition to allowing customisation to enable access to a wider audience, people with autism could customise the system to mimic their individual experiences and from this we could draw conclusions with machine learning techniques on how people with autism generally experience sensory difficulties without needing required skills to create their own scenarios if they could adjust settings to make it more realistic to how it is for them and have this information sent for analysis.

\subsection{Difficulty setting}
People could answer questions depending on their autism experience and their gaming experience and a difficulty would be set. I feel this would offer something better than simply "easy", "medium" or "hard". Formatives were mostly with people whom had gaming experience so more tests needs to be done on people whom do not.

\subsection{Further research}
A questionnaire for users after playing to see which aspects are being successfully conveyed with more stories or vignettes or an entirely different approach. 


\section{Public response}
\label{sec:publicresponse}
The feedback so far has been overwhelmingly positive. Inspite of only emailing around a few people in attempt to get feedback online and not advertising on youtube or twitter but simply asking others to forward information to others they may know, the website reached 1500 hits in the space of a few days. This result alone I feel demonstrates the wide need and interest and need of such a project. 

\begin{quote}
"Watching the video I felt empowered and more comfortable voicing myself. Years ago I was institutionalised. I tried to explain that I felt I was being sucked into an alternative universe and they took that as I was crazy but I wasn't it was the feeling caused by a sensory overload and no-one understood. I never knew that at the time and I never knew how to explain it. If I had something like this back then I feel it would have really helped. All I want to do now is go and give this to them and say "Look! This is what was happening!" - Anonymous feedback from an adult with autism.
\end{quote}

\begin{quote}
"I think parents, or people with no idea about sensory issues would find it very informative including some affected people who have an acceptance of their condition and are beginning to learn about it.” - Julie Brown. Director of Kids can achieve(on whom she felt the simulator would be useful for) 
\end{quote}

\section{Conclusions}
Goals 1-3 to have been clearly met, however, it cannot be convincingly concluded that the 4th and more research is required. 

This was an extremely large project that had a lot of trepidation on whether it was possible given the time restraints. It was also a project that required multiple roles; a researcher, a graphics designer and artist to create the 3d models and images, and a programmer; skills that are separate in the game programming industry. Some more ambitious initial plans were unfortunately dropped however, the system developed is a solid platform with a modular architecture and is already in a state in which people could design their own scenarios and tasks now; although better documentation is definitely required. Very fortunate position to create this with having family members on the autistic spectrum and supervisors with a lot of experience in designing learning adaptive environments and specifically autism technology.

From a virtual reality conference held in Edinburgh by the Albertus Institute(January, 2014) there was quite a lot of negativity surrounding the use of virtual reality and the effects this can have on personhood such as becoming too attached to in-game characters. The risk of this in relation to this project is that if the users experience too many negative attributes of autism of take the simulator too literally they may leave feeling pity. However, overall I feel this project has given some indication that virtual reality and games really do have a lot of positives to offer; there was little research documenting the success of using simulations to aid understanding of neurological conditions, but this project has shown that visual simulations have at least the potential to be an extremely successful tool.