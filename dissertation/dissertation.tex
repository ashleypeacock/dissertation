\documentclass[11pt]{report}
\begin{document}

\title{Autism Simulator}
\author{Ashley Peacock}
\maketitle
\tableofcontents
\chapter{Introduction}

\section{Selecting a project}


\chapter{Literature review}
\section{About autism}

Autistic Spectrum Disorder is a lifelong condition which affects how an individual communicates and may perceive the world around them \cite{nas}. The severity and type of difficulties vastly differ and are unique to each individual. Autism is currently diagnosed by the presence of atypicalities in three domains: social imagination, social communication, social interaction, collectively known as the 'Triad of impairments', there are also nondiagnostic but highly prevalent features such as sensory abnormalities, information processing difficulties, unusual fears and prosopagnosia. With some of the disadvantages that people with autism face, come reported strengths too; for example a talent for spotting details or having a memory for encyclopaedic knowledge of their 'special interests'. 


\subsection{Triad of Impairments}

There are three key areas of difficulty that people with autism share.

\subsubsection*{Social communication}
People with autism have difficulty with verbal and non-verbal language such as body language or tone of voice. Language tends to be interpreted literally and thus metaphors, sarcasm and jokes can be difficult to understand\cite{nas}. An example of literal interpretation is where a person with autism misunderstood the question "What's up?" and proceeded to look up at the ceiling. Other communication difficulties include echolalic language(repeating language said to them) or speaking excessively about their 'Special interests' without possibly detecting that the other party are bored. Although people with autism will usually understand what is being said to them they may prefer to use visual symbols such as PECS(Picture Exchange Communication System). 

Due to language being interpreted literally, it is important that language communicated is clear, concise and unambiguous, one of the needs the public were most unaware of\cite{autismmisconception}.

\begin{quote}
Most things I take at face value, without judgement or interpreting them. I look at them in a concrete, literal and very individual way. \cite{olgab}
\end{quote}

\subsubsection*{Social interaction}

\begin{quote}
Autistic people have to understand scientifically what non-autistic people already understand instinctively 
- Mark Segar, Autistic Survival Guide.
\end{quote}

Many people with autism have difficulty giving eye contact, one person described eye contact as "physically painful". By not giving eye contact, it may cause social queues such as facial expressions to be missed, potentially leading to inappropriate responses. Lack of eye contact could be perceived as rude or not paying attention to the speaker, possibly causing unintentional offence. Other social interaction difficulties include reduced understanding of social rules\cite{nas}, for example why people say 'thankyou'.


\subsubsection*{Social imagination}
Social imagination deficits result in difficulties 'Putting themselves in another person's shoes', also known as 'Theory of mind'. Other resulting difficulties include problems predicting events or identifying possible dangers such as running across a road and consequently, new situations can be difficult.\cite{nas}

Social imagination difficulties can make it hard for a child with autism to engage in imaginative play, preferring to act out scenes from films identically which can make it difficult for other children interacting with them if they prefer to deviate from a plot.\cite{nas}

\subsection{Information processing}

It is suggested that people with autism process information holistically, a theory known as Gestalt perception. Processing in this way is posited to cause a fragmented perception, a result when there is not enough resources available to process all information at once and draw connections. "I had always known that the world was fragmented. My mother was a small and a texture, my father was a tone, and my older brother was something, which was moving about" \cite{williams1992}. As a result of fragmented perception, it may take someone with autism longer to adjust to their surroundings, objects when looked upon from a different angle may look entirely different and unfamiliar. One small change to the environment can therefore cause a large amount of distress.

Difficulties filtering information also cause problems differentiating between background and foreground noise. In a room with many people talking, it may be hard to tune into an individual conversation. 

It is argued that people with autism perceive the world more accurately\cite{bayes} because their inferences are less dependent on previous experience\cite{bayes}. Delacato(1994) reported that children with autism who are hyper-visual are not fooled by optical illusions\cite{bayes} as most people filter information based on what they are expecting to see, using this to fill in the gaps in and are thus prone to these optical illusions. 

\subsubsection{Sensory processing}

While social and communication difficulties are core symptoms and most commonly associated with autism in the public view, "Many people with Asperger syndrome/High functioning autism define their sensory processing problems as more disabling than the deficits in communication/social behaviour\olga{b}. Sensory processing differences in autism are highly reported, 81\% of respondents reported differences in visual perception, 87\% in hearing, 77\% in tactile perception, 30\% in taste and 56\% in smell \cite{olgab}. Senses play a vital role in how we model and perceive the world around us so if one senses the world in a differently, their view and behaviours will also be different. 

Senses in autism can be hyper(more sensitive), hypo(less sensitive), agnostic or fluctuate between hyper and hypo\cite{bayes}. As with all areas of autism, sensory atypicalities differ and are unique to each individual, however, these fluctuations make it an area of particular challenge for carers and for a person with autism to identify or predict troubling sources before they occur and can be compared to a 'FM radio that is not exactly tuned on the station when you are driving down the freeway. Sometimes the world comes in clearly and at other times it does not" \cite{olgab}.

Catering with for the many different sensory needs for many different children can be very demanding. In the classroom for example if a child is hypo-visual and feels a need to stimulate their visual senses by constantly switching on and off a light, in contrast to another child in the class being hyper sensitive, the result could lead to a sensory or information overload(this was commented on in one of the interviews from the teacher...).

Below are some examples of the effects someone with autism may experience depending on the state of their sensory channel:

\begin{table}
    \begin{tabular}{| l | p{5cm} | p{5cm} |}
    \hline
    Sense channel & Hyper                                                                                                                      & Hypo                                                                   \\
    \hline
    \hline
    Vision        & Vision may be magnified                                                                                                    & Attracted to light or fascinated with bright colors                    \\
    \hline
    Auditory      & Sounds are amplified. Temple Grandin a write with autism described her ears as like 'microphones'                          & Is attracted to sounds/noises                                          \\
    \hline
    Tactile       & Clothes may hurt. One person with autism described clothe labels as feeling like 'barbed wire'. May not like being hugged. & Enjoys being hugged or seeks pressure by crawling under heavy objects. \\
    \hline
    Taste/Smells & Smells or texture of foods may be intolerable. & Mouths and licks objects \\
    \hline
    Vestibular & Difficulty with walking or crawling on uneven or unstable surfaces. & Spins, runs round and round, rocks back and forth \\
    \hline
    \end{tabular}
\end{table}

Sensory processing patterns can be categorised into four-types\cite{sensory_leisure} (below is probably not much use at the moment, but useful for later justifying the game character's characteristics and responses to the environment):

\begin{enumerate}
\item Sensory avoidance pattern: Low sensory threshold. 
\item Sensory seeking patterns: A high sensory threshold and make seek out stimulus.
\item Sensory sensitivity patterns: Low thresholds and may respond to stimulus more intensely or for a longer period of time.
\item Low registration: High sensory threshold, may appear not to detect incoming sensory information and also show a lack of responsiveness.
\end{enumerate}

Correlation between sensory difficulties and difficult temperament characteristics such as activity level, adaptability to changing context, quality of mood, threshold of responsiveness, intensity of reaction and persistence\cite{temperament}. 

\subsection{Resulting behaviours}


\subsubsection{Sensory/Information overload}

When the amount of information required to be processed comes in high abundance too quickly it can result in someone with autism experiencing an 'Information' or 'Sensory' overload.

Overloads can also cause the senses to become agnostic, causing their system to shut down completely, effectively being able to see but being "mind-blind", "understanding-blind" or "mind-deaf".

Donna Williams reports that "sensory overload caused by bright lights, fluorescent lights, colours, and patterns makes the body react as if being attacked or bombarded, resulting in such physical symptoms as headaches, anxiety,
panic attacks or aggression"\cite{bayes}.

The resulting behaviours can again differ for each individual but in general there appears to be three responses:

\begin{enumerate}
\item Fight child may appear to throw a tantrum, hit out at others around them or engage in self injurious behaviours. 
\item Flight child make run away without consideration of the dangers around them.
\item Withdrawn child may become completely withdrawn. Cover eyes/ears and not respond to the environment around them. 
\end{enumerate}


\subsubsection{Meltdowns}
Meltdowns are the result of severe stress or becoming emotionally overwhelmed. Sensory overloads can thus progress in meltdowns if they are not dealt with appropriately. Meltdowns can manifest in different ways. A person with autism experiencing a meltdown may feel themselves enter a state of 'fight or flight', or they may become severely withdrawn.

\subsubsection{Repetitive behaviours}

With processing information holistically, one slight change to an environment can make it appear entirely different.

Sensory processing atypicalities are posited as a cause for repetitive behaviours and a need for routine. With an ever changing environment, routine can be their only sense of familiarity and reassurance. Interestingly it is reported that people with autism can have more problems with small changes than with big ones, for example being able to cope better going to a completely new environment over furniture changes in the house\cite{olgab}. 93\% of children with autism were reported to be distressed by change \cite{fears}   

If senses or part of the brain are not working, a child may try to stimulate them by 'hand flapping', turning lights on and off.

\subsubsection{Unusual fears}
It was found that 40\% of children with autism had unusual fears in comparison to 0-5\% of typical children, the vast majority of these were made up of mechanical objects. Children with autism have higher levels of anxiety than typical children\cite{fears} and increased anxiety from being faced with more fears on a day to day basis will only increase this and further impact on functioning. For example, not leaving the house because it's cloudy, or not taking a shower because of the noise from the drain, not going to school due to being afraid the fire alarm will sound. The top five reported were toilets, elevators, vacuum cleaners, thunderstorms, tornadoes\cite{fears}.

\section{Impact and Prevalence}
Although it is difficult to calculate an exact figure, it is estimated in the UK that there is approximately 1.1\% of the 
population with Autism based upon the 2011 census\cite{nas}. This figure appears to be rising across the globe as awareness and understanding of the condition increases alongside broadening criteria\cite{increasingprevalence}, for example the inclusion of Aspergers syndrome which has only been a formal diagnosis since 1990. Early estimates of autism spectrum disorders were identified as less than 10 in 10,000, this has grown to a current estimation of 110 in 10,000 in the USA \cite{increasingprevalence}.

It is estimated that only 22\% of teachers have been trained specifically in autism and the majority of training given is typically one to four hours. 54\% of all teachers in England do not feel they have had adequate training to teach children with autism.\cite{statsandfacts} 30\% of parents of children with autism in mainstream education are satisfied with the level of understanding of autism across the school\cite{nasschool}. 23\% of parents are dissatisfied with SENCO's level of understanding of autism. 

It is estimated that around 40\% of children with autism have been bullied at school. 1 in 5 children with autism have been excluded from school \cite{nasschool} and only 24.4\% of pupils with autism achieved 5A*-C GCSEs in 2010/2011 in comparison to 58.2\% of the overall population\cite{statsandfacts}, a surprising figure owing that people with autism are deemed to have average to above average intelligence which indicates difficulties at school may be a reason for not reaching their potential.

\begin{quote}
Danny would not have been excluded if the school had understood the difference between 'normal' behaviour and Aspergers syndrome. They inflamed situations because they didn't understand that my son finds physical contact, or being touched by teachers, really difficult \cite{nasschool}
\end{quote}

\begin{quote}
If I could make one change...I would ensure compulsory, thorough training about autism and how it affects learning is given to all school staff. \cite{nasschool}
\end{quote}

\section{Public perception of autism}
Although there is some level of awareness of autism in the public domain, there is still much left to be desired.  From a survey carried out by the National Autistic Society, 92\% of respondents had heard of Autism but only 48\% had heard of Aspergers syndrome which has less obvious difficulties and is consequently regarded as a 'hidden disability'. Most were able to identify key characteristics such as difficulty communicating or making friends whilst other common characteristics such as a need for 'clear unambiguous instructions and sensory hypersensitivity to noise were less known\cite{autismmisconception}. 

// find sources on public interaction with people who have autism.

\begin{quote}
If I could make one change... every person who comes into contact with my daughter would have some form of training in autism.\cite{nasschool}
\end{quote}


\section{Previous work}

\subsection{Education software}
How other education software can be used to help children with understanding autism or learning. 

\subsection{Other simulators}
Include the one that was released this year and possibly other similators that have been used to convey other disabilities if I can find them...

\section{Other}
Just for now, some information that might be of use to put somewhere, not sure where.

"if you deal with 'challenging behaviours' in autism, do not focus on the iceberg; do understand the underlying causes of the behaviours and try to develop an approach not based on symptoms but on prevention. Challenging behaviours are caused by problems of communication, social understanding, by different imagination, by sensory problems...Therefore try to understand autism 'from within'. It is easier said than done, because it requires an enormous effort of imagination: we need t learn to put ourselves in the brains of autistic people and then we will understand better through their eyes the obstacles in their attempts to survive among us" - Theo peeters \cite{olgab}

\begin{quote}
It doesn't appear that mainstream teachers have had access to training. The fundamental issues relating to communication, behaviour and language disorder continue to be misinterpreted as 'bad behaviour', 'not listening' and so on.\cite{nasschool}
\end{quote}


\begin{quote}
Action to increase understanding of autism across the whole school and to provide support with social activities can make a huge difference to whether a child with autism feels included at school.\cite{nasschool}
\end{quote}



\chapter{Design process}
\section{Interviews}

\section{Game design}
\subsection{House design}
House design choices. 
\subsection{Character}
\subsubsection{Autism aspects to convey}
\subsection{Design of sensory system}
\subsection{Story boards}

\chapter{Prototype}

\section{Implementation}
What was in the prototype.
\section{Evaluation}
\subsection{Expert feedback}
\subsection{User feedback}
\section{Improvements}

\chapter{First version}

\chapter{Final version}

\begin{thebibliography}{9}

\bibitem{findref}
Find a reference

\bibitem{increasingprevalence}
Johnny L. Matson, Alison M. Kozlowski
The increasing prevalence of autism spectrum disorders. Research in Autism Spectrum Disorders(2011)

\bibitem{nas}
National autistic society. www.autism.org.uk

\bibitem{statsandfacts}
Ambitious about autism. www.ambitiousaboutautism

\bibitem{nasschool}
Make school make sense. Autism and education: the reality for families today. National Autistic Society, 

\bibitem{autismmisconception}
Autism misconceptions. NHS

\bibitem{olgab}
Sensory Perceptual Issues in Autism and Aspergers syndrome(2003). Bogdeshina, O

\bibitem{fears}
Unusual fears in children with autism(2012). Susan Dickerson Mayes, Susan L. Calhound, Richa Aggarwal, Courtner Baker, Santosh Mathapati, Sarah Molitoris, Rebeccas D. Mayes.

\bibitem{temperament}
Sensory correlates of difficult temperament characteristics in preschool children with autism(2012). I-Ching Chaung, Mei-Hui Tseng, Lu Lu, Jeng-Yi Shieh

\bibitem{bayes}
When the world becomes 'too real': a Bayesian explanation of autistic perception(2012). Elizabeth Pellicano and David Blurr.

\bibitem{williams1996}
Autism: An inside-out approach(1996). Williams, D

\bibitem{williams1994}
Somebody Somewhere: Breaking Free from the World of Autism(1994). Williams, D.

\bibitem{williams1992}
Nobody no-where(1992). Williams, D

\bibitem{sensory_leisure}
Sensory processing abilities and their relation to participation in leisure activities among children with high-functioning autism spectrum disorder. Hochhauser, M. Engel-Yeger, B.


\end{thebibliography}

\end{document}