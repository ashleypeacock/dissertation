\documentclass[11pt]{report}
\usepackage{fullpage}
\usepackage{graphicx}
\usepackage{float}
\restylefloat{table}
\usepackage{url}
\usepackage[toc, page]{appendix}
\usepackage{listings}
\begin{document}
\title{Developing a tool for teachers to increase awareness and understanding of Autism}
\author{Ashley Peacock}
\maketitle
\begin{table}[H]
    \begin{tabular}{| p{6cm} | p{6cm} |}
    \hline
    \textbf{Vignette A} & \textbf{Vignette B} \\                                                                                                                                                                                     
    \hline
    \hline
    Johnny is eleven years old and has autism. He is in your mainstream class and today you are working on fractions. It is a lovely sunny day and outside someone is mowing the football pitch.  Johnny normally enjoys maths but today he is fidgety and restless.  When you ask him why he doesn't respond.  Eventually he jumps up and leave the classroom without asking permission. & Emily is nine years old and has autism. She is in your mainstream class and today they are doing group projects on using money. Emily's group are annoyed that she keeps taking the coins they are working with and spinning them, and they come to you to complain. \\ 
    \hline
    \end{tabular}
\end{table}

\begin{table}[H]
    \begin{tabular}{| p{6cm} | p{6cm} | p{1cm} |}
    \hline
    \textbf{Vignette A response} & \textbf{Vignette B response} & \textbf{Score} \\                                                                                                                                                                                    
	\hline
	Ask the children sitting near Johnny to go and bring him back in & Tell the children to work it out between themselves & 0 \\
 	Set Johnny extra maths homework to make up for the lessons he missed & Send Emily outside for disrupting the class & 0 \\
 	Close the classroom windows & Explain to Emily that the spinning is making it difficult for her group & 2 \\
	Find Johnny and ask him what is wrong & Ask Emily why she is spinning the coins & 2 \\
	Call Johnny’s parents and find out what he ate for breakfast & 	Call Emily’s parents and ask them whether she gets any pocket money & 0 \\
	Send Johnny to the Headmaster for punishment & Tell Emily to stop spinning the coins and focus on her work & 0 \\
	Offer Johnny a chance to work on his own & 	Offer Emily a chance to work on her own & 2 \\
	Make Johnny stay inside over lunch to catch up & Take all the coins away from that group &  0 \\
	Go outside and ask the gardener to stop mowing the lawn & Give Emily something else she can spin & 3 \\
	Send Johnny to the guidance counsellor & Send Emily to the guidance counsellor &  0 \\
	Find Johnny and give him a hug & Tell the other children not to bother Emily because she is special & 0 \\
    \hline
    \end{tabular}
\end{table}

\end{document}