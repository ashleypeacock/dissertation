\chapter{Formative evaluation: Prototype}
The current version of the simulator has been evaluated in two settings. The first was the presentation of the simulator to LAERLab. The second was a questionnaire sent to some parents/family members of children with autism and adults with autism.

\section{Expert feedback: LAERLab}
The LAER group consists of students and academics with an interest in the field of accessible and educational technology design and 10 members provided comments and feedback.

A short video demo was presented and narrated showing exploration of the environment followed by one task being completed (obtain a drink from the kitchen). The demo included sensory overload effects, meltdown and information processing delay concepts. (**linux died on my laptop that I had used on the day and I'm pretty sure don't have this video!! Unless I sent it to other people via email. Should I just make another? )

\subsection{Results}

\begin{enumerate}
\item Lab members disliked the pop-up information boxes as they felt it interrupted the simulation. It was suggested information may be better put as a sidebar or bottom bar.
\item Made clear that the simulation information provided in the description boxes are just samples of things known to present problems to children with autism and do not represent all children with autism.
\item Idea put forward to allow the user to select the different issues relating to autism and then to test the consequences of this in the environment. 
\item Demo was a bit too quick, one of the problems being the camera is controlled by the mouse and the video was recorded on a laptop (so camera motion was not smooth). The general controls are quite sensitive.
\item Suggestions for tactile sensory when touching objects(create a slightly red blur around the edges as is common in games).
\end{enumerate}


\section{User feedback}
A video demo was briefly put on-line with verbal narration of what was happening in the simulator. No information on goals or target audience was given as users were asked whom they felt it would be useful for. The questionnaire which was completed by 1 family member of someone with autism, 3 parents of children with autism and 3 adults with autism. It contained both qualitative and quantitative questions. Some users chose not to respond using the questionnaire but gave comments. 

*** Reminder to self. Think I saved a massive amount of feedback on this where people didn't use the questionnaire; get from desktop.

\subsection{Results}

\begin{table}[H]
\caption{Questionnaire results. Participants responded giving an answer between 1 and 5. 5 being the highest rating}
\begin{tabular}{| p{9cm} | c c c c c |}
\hline
\textbf{Question} & 1 & 2 & 3 & 4 & 5 \\
\hline
How likely would you be to play the simulator? & 0 & 0 & 1 & 2 & 4 \\
\hline
How much do you think playing the simulator could help you understand the behaviour of a child with autism better? & 0 & 0 & 0 & 2 & 5 \\
\hline
How much did you like the visual effects? & 0 & 0 & 1 & 3 & 3 \\
\hline
How much did you like the graphics? & 0 & 0 & 2 & 2 & 3 \\ 
\hline
\end{tabular}
\end{table}

The qualitative questions asked were:
\begin{enumerate}
\item What is your experience with autism? (for example if you are a parent, professional, family member or someone with ASD yourself)
\item Who would you think the simulator would be useful for?
\item How accurate do feel the representation of a sensory overload is? Please include any additional comments.
\item Do you have any other scenarios, examples or information(that could go in description boxes for example) that could be offered that you think may be useful for others?
\end{enumerate}

Some of the responses to qualitative questions can be seen below, some participants chose not to give qualitative answers and only responded to quantitative questions:

*** appendix with brief summary and analysis?

What is your experience with autism? (for example if you are a parent, professional, family member or someone with ASD yourself)
\begin{table}[H]
    \begin{tabular}{| p{3cm} | p{12cm} |}
    \hline
     Participant & Comments \\ \hline
     1 & Parent of two ASD children. I also have ASD traits which may be aspergers. \\ \hline
     2 & I have a son now aged 15yrs diagnosed at Peterborough aged 3yrs 9 mnts by child development specialist and her team. Who gave enormous help in getting Tom into nursery and then school and regular assessments. We moved back to Ireland my home place when Tom was 6yrs and have had no help since(mistake). Tom has not been in school for more than 3 weeks since age 12yrs due to sensory issues and has recently started taking prozac for anxiety, which has helped enormously! Your simulator is a brilliant idea which will be a major asset to both parents educators struggling to understand a childs condition. There are so many scenes and scenarios that this can cover and it could be extended to outside the home to school and outside spaces! One example - I remember when Tom was about 5/6, we were at the park and it started raining so we headed home as we rounded the last corner Tom stopped and wanted to go back to start of corner! This was repeated 5 times and he was getting more and more annoyed starting to scream. He was insisting we go back again and it was pouring rain, in the end I said thats it and I ran down the road to the house and peered out and he eventually tottered down the road! But it took me ages to figure out what the problem was until I realized it was that every time we started to go round the corner a car came and the engine noise affected/disturbed his action/experience/sensation of something we take for granted and would not notice. A video simulation depicting such sensory agitators would make an enormous difference to carers and how they react to situations. Hope you found my reply helpful I have many examples such as how buttons on his school clothes caused meltdowns every-morning or how the simple task of changing a nappy was hell until we learned to talk the child through in stages. Best regards  \\ \hline
     3 &  I'm autistic  \\ \hline
    \end{tabular}
\end{table}

Who would you think the simulator would be useful for?
\begin{table}[H]
    \begin{tabular}{| p{3cm} | p{12cm} |}
    \hline
     Participant & Comments \\ \hline
     1 & Teachers, teaching assistants, health care professionals, families. \\ \hline
     2 & - \\ \hline
     3 & People who don't know much about autism - eg a teacher with an autistic student \\ \hline
    \end{tabular}
\end{table}

How accurate do you feel the representation of a sensory overloads are? Please include any additional comments.
\begin{table}[H]
    \begin{tabular}{| p{3cm} | p{12cm} |}
    \hline
     Participant & Comments \\ \hline
     1 & I have sensory difficulties so I felt uncomfortable when it became brighter plus the talking. That is me though and I have these sorts of experiences a lot. Not sure how it would feel for someone without sensory issues. \\ \hline
     2 & I think the reactions in the kitchen is good, maybe for other subtle agitators thought clouds could be used! Plus I think the simulator could be used in schools etc in educating young people, but I might prefer to maybe watch as a video if it were possible to have both formats!! \\ \hline
     3 & A lot more severe than how I experience it, but then I've got milder issues than many I know. But the only kids I know who'd go into a full meltdown that quickly from being near a washing machine would be low functioning (eg minimally verbal) so you may want to back it off a bit. (A fire alarm, on the other hand, would get that strong a reaction even from me.) I really like the effect of the swirling options. It's really unexpected, how well it captures what trying to speak when overloaded is like. I'd never have thought of that analogy, but it's perfect. \\ \hline
    \end{tabular}
\end{table}

Please let me know if you have any other general feedback on the simulator. E.g improvements, suggestions or what you would change.
\begin{table}[H]
    \begin{tabular}{| p{3cm} | p{12cm} |}
    \hline
     Participant & Comments \\ \hline
     1 & Felt more could be done with illustrating the pain of noise. Wasn't sure if the simulator did this; I was watching it rather than listening:( Hard for me to do both online. \\ \hline
     2 & - \\ \hline
     3 & Would it be possible to allow the user to chose between 1st and 3rd person mode? I find 1st person really hard to navigate in. \\ \hline
    \end{tabular}
\end{table}

Do you have any other scenarios, examples or information (that could go in the description boxes for example) that could be offered, that you think may be useful for others?

\begin{table}[H]
    \begin{tabular}{| p{3cm} | p{12cm} |}
    \hline
     Participant & Comments \\ \hline
     1 & Try to include some of the good parts of autism as well. For example, I really enjoy sparkly things. If you make him like sparkly things too, maybe you could highlight the sparkly things and make them look extra awesome to show that.  \\ \hline
     2 & - \\ \hline
     3 & I think if you could use the thought clouds as the avi walking around he/she can be expressing discomfort ie hearing unwanted noises etc my son used to be terrified of the vacuum cleaner! So many little things too! \\ \hline
    \end{tabular}
\end{table}


\section{Conclusions}

Following constructive feedback from these sources, the following amendments to the simulator have been selected:

Description boxes: the HUD can to be changed such that information on the environment does not affect game play. Descriptions could be implemented instead as 'thoughts' which would be displayed at the bottom of the screen and when looking at certain objects. An alternative would be to have a 'Task' which would simply enable no other game play apart from exploring and obtaining information on surroundings, whilst still retaining the pop-up boxes. Images could also be included to give better impact for example when explaining that clothe labels can feel like barbed wire to someone with autism an image of barbed wire and the information can be shown.

Meltdown system: contentment needs to simply reduce less when there are sensory problems around the environment although certain sound objects or lights may need to give more impact than others. I.e a fire alarm causing more problems than the sound of the tv. An internal action such as 'close eyes' might also be used. 

Game play: remove mouse for controlling the camera and allow simple arrow direction keys to be used. Also adjust the player movement so it closer resembles walking. Additionally implement an internal action 'run' to allow for faster movement although one suggestion would be for this to increase the chance of bumping into objects of falling over. 

Interface: User instructions for the start of the game need to be implemented as well as menu's and options for enabling or disabling certain sensory effects as viewers with autism commented they found it difficult to watch as it was causing them to have a sensory overload. 