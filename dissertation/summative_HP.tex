
\chapter{Summative evaluation}
The goals of this summative evaluation were a) to discover if teachers can acquire a greater understanding or awareness of autism sensory difficulties by playing the simulator and b) to draw conclusions as to whether this tool could be successfully used as a tool for teacher training. 

Two similar evaluations were conducted in person and online, with the differences in methodology and participants explained in the alternative methods sections. The in-person evaluations were conducted before the online; some improvements were made in between to ensure that the on-line participants could play with minimal materials as the researcher would not be there to answer questions on game play or give help. 


\section{Methods}
Two vignettes were created in consultation with an expert in Education, Developmental Disabilities, Autism and Technology. Each entailed a story of a potential classroom encounter with a child whom has Autism, as well as potential responses to the situation. Scores were assigned to each response, zero being given for an in appropriate response, 2 for a good response and 3 for the best response.

Participants were asked to read the first vignette, tick three options indicating how they best felt they would respond in the situation. In addition, they completed a consent form and background questionnaire on their prior autism experience, feelings towards managing children with autism and computer game experience. Following this they were asked to play the explore mode of the autism simulator, followed by the mission mode, and after playing the simulator participants were asked to complete a second vignette. There were differences in the versions of the simulator used in person and online. These are described below. 

Scores were assigned to the vignette responses with matched answers and points awarded across the two different vignettes. The order that participants completed the vignettes was alternated to reduce the possibility that one was more difficult than the other, affecting the data. In addition, the responses and marks attributed were in the same order in both cases. The full background questionnaire, consent form and vignettes used can be seen in the appendix.

\begin{table}[H]
    \begin{tabular}{| p{8cm} | p{8cm} |}
    \hline
    \textbf{Vignette A} & \textbf{Vignette B}                                                                                                                                                                             \\    
    \hline
    \hline
    Johnny is eleven years old and has autism. He is in your mainstream class and today you are working on fractions. It is a lovely sunny day and outside someone is mowing the football pitch.  Johnny normally enjoys maths but today he is fidgety and restless.  When you ask him why he doesn't respond.  Eventually he jumps up and leave the classroom without asking permission. & 
	Emily is nine years old and has autism. She is in your mainstream class and today they are doing group projects on using money. Emily's group are annoyed that she keeps taking the coins they are working with and spinning them, and they come to you to complain.\\ 
    \hline
    \end{tabular}
    \caption{Table of the two vignettes used}
\end{table}

\begin{table}[H]
    \begin{tabular}{ |p{1cm} | p{6cm} | p{6cm} | p{1cm} |}
    \hline
    No & \textbf{Vignette A response} & \textbf{Vignette B response} & \textbf{Score} \\                                                                                                                                                                                    
	\hline
	1 & Ask the children sitting near Johnny to go and bring him back in & Tell the children to work it out between themselves & 0 \\ \hline
 	2 & Set Johnny extra maths homework to make up for the lessons he missed & Send Emily outside for disrupting the class & 0 \\ \hline
 	3 & Close the classroom windows & Explain to Emily that the spinning is making it difficult for her group & 2 \\ \hline
	4 & Find Johnny and ask him what is wrong & Ask Emily why she is spinning the coins & 2 \\ \hline
	5 & Call Johnny�s parents and find out what he ate for breakfast & 	Call Emily�s parents and ask them whether she gets any pocket money & 0 \\ \hline
	6 & Send Johnny to the Headmaster for punishment & Tell Emily to stop spinning the coins and focus on her work & 0 \\ \hline
	7 & Offer Johnny a chance to work on his own & 	Offer Emily a chance to work on her own & 2 \\ \hline
	8 & Make Johnny stay inside over lunch to catch up & Take all the coins away from that group &  0 \\ \hline
	9 & Go outside and ask the gardener to stop mowing the lawn & Give Emily something else she can spin & 3 \\ \hline
	10 & Send Johnny to the guidance counsellor & Send Emily to the guidance counsellor &  0 \\ \hline
	11 & Find Johnny and give him a hug & Tell the other children not to bother Emily because she is special & 0 \\ 
    \hline
    \end{tabular}
    \caption{Table of the responses and associated scores for each response}
\end{table}

\subsection{In-Person Participation}
Participants were recruited from the university of Edinburgh's School of Education. An email was sent and participants were asked to email myself with their preferred time. 8 people responded with 2 being unable to find a suitable time. Evaluations were conducted in ``The Infant Lab�� in the School of Education and lasted approximately 45 minutes. It was hoped to find more first year or younger teachers with less knowledge on autism however most participants came from a background of prior knowledge on autism. Following positive formative evaluation, the simulator trial focused on the morning routine, as there was thought to be more than enough content to aid teaching of autism, and the afternoon mission was not sufficiently well tested. Participants were permitted to ask questions on game-play if they encountered difficulty and any questions were noted, though few were asked.

Participants were asked to complete the background questionnaire and first vignette, and were given little information on the simulator except ``you play as a child with autism and get to experience some of the difficulties through their eyes��. After playing with the simulator, participants were asked to complete the second vignette. The researcher would then answer any questions that participants had on autism, or on the simulator itself, and take feedback. 

\subsection{Online Participants}

The background survey and two vignettes were put on-line along with a video tutorial which contained basic information of how to use the simulator. This highlighted the fact that no two people with autism are the same, as can be seen on http://autism-simulator.com. The simulator was available for downloaded rather than as a java applet due to problems with obfuscation. 

Following the in-person evaluation sessions, some improvements made before putting the simulator online: 

\begin{enumerate}
\item The tool-tip which reveals what actions are available was still not obvious. This was moved to below the cross hair and thus directly into the users line of sight.
\item Thoughts were made larger.
\item Some bugs were fixed. 
\item Starting point in the bedroom was moved to a more central location; before, it had started close to the door and one participant was confused about their surroundings for a few moments.
\item Time for the morning routine now counted down 50\% slower.
\item Changes to the start screen.
\end{enumerate}

Teachers were emailed from a previous school attended by the researcher, and they in turn attempted to find a few other teachers from their school. Teachers were given a link to the website with the following instructions

\begin{enumerate}
\item \textbf{Before} you play the simulator fill out \underline{this} preliminary questionnaire first
\item \textbf{Before} you play the simulator fill out either Vignette A or Vignette B first. I need to have a general equal amount of people doing A or B first.
\item Watch the video tutorial
\item Play the explore mode on the simulator and when you feel ready, play the "Mission mode" of the simulator. If you come across any bugs, please report in the final questionnaire. If you end up completely stuck shut down and restart the simulator by pressing "Esc" key.
\item \textbf{After} you have played the simulator fill out the other Vignette you did not complete first. If you don't I cannot use your data!
\item Fill out the final questionnaire for additional comments and feedback(this part is optional)
\end{enumerate}


\section{Results}


Below is a summary of total scores and improvements. Full results of answers can be seen in the appendix. 

\begin{table}[H]
    \begin{tabular}{| p{1cm} | p{2cm} | p{2cm} | p{3cm} | p{3cm} | p{3cm} |}
    \hline
    \textbf{No} & \textbf{Total before} & \textbf{Total after} & \textbf{Improvement} & \textbf{First Vignette} & \textbf{Second Vignette} \\                                                                                                                                                                                    
	\hline
	1 & 6 & 7 & 1 & A & B \\ \hline
	2 & 7 & 7 & 0 & B & A \\ \hline
	3 & 6 & 7 & 1 & A & B \\ \hline
	4 & 7 & 7 & 0 & B & A \\ \hline
	5 & 5 & 5 & 0 & A & B \\ \hline
	6 & 7 & 7 & 0 & B & A \\ \hline
	
	8 & 7 & 7 & 0 & B & A\\ \hline
	9 & 7 & 7 & 0 & A & B \\ \hline
	10 & 6 & 5 & -1 & B & A \\ \hline
	11 & 4 & 7 & 3 & A & B \\ \hline
	12 & 7 & 7 & 0 & A & B \\ \hline
	13 & 4 & 7 & 3 & A & B \\ \hline
    \hline
    \end{tabular}
    \caption{Summary of results. Participants 1-6 were in person. Participants 8-12 were online. Participant 13 was in person but completed questionnaire online. 6 online participants provided incomplete data and are omitted from the analyis}
\end{table}

There are no significant differences between the scores on vignettes before and after using the simulator, though all scores but one either improved or remained the same. 

%% Do you have any performance data? Did they all complete the mission? How
%% many meltdowns did each have? DO you have any of that data?


From the data we can see that there were either improvements of understanding, or in the case where participants already had maximum scores, their score remained the same. Participant 5 only gave two responses, instead of three, but they did receive the maximum score for both.

Most participants came from a very knowledgeable backgrounds in autism and this may be why improvements are only slight, due to a ceiling effect. On-line participants were from a variety of backgrounds rather than specific autism backgrounds and this shows in their initial scores. 

For the one candidate whom showed a reduction in overall score, their initial and latter response can be highlighted below:

\begin{table}[H]
    \begin{tabular}{| p{6cm} | p{6cm} |}
    \hline
    \textbf{Response before} & \textbf{Response after}  \\                                                                                                                                                                                    
	\hline
	\hline
	Explain to Emily that the spinning is making it difficult for her group & Set Johnny extra maths homework to make up for the lessons he missed \\ \hline
	Ask Emily why she is spinning the coins & Close the classroom windows \\ \hline
	Offer Emily a chance to work on her own & Go outside and ask the gardener to stop mowing the lawn \\ 
    \hline
    \end{tabular}
\end{table}

From this we can see that their sensory knowledge appeared to have been improved with both related responses given whilst doing the potentially harder second vignette. It might be argued that although the participant�s overall score reduced, their knowledge in sensory difficulties in autism increased. 

%% comment: �Thus it can be concluded� is too strong here

\section{General discussion}

The first person who took part in the summative evaluation had a hearing impairment and arrived with an interpreter. Owing to a quite a substantial aspect of the simulator experience being the heightened sounds, it was uncertain whether the full experience would be successfully conveyed. The participant finished the mission in the bedroom, unable to complete the morning routine, commenting ``It's less stressful to be in here��. Afterwards the participant asked questions such as, ``If someone with autism comes to a meeting would you suggest I offer them a quiet room?��. This would be a recommendation and such questions indicate that, even without having the full experience, the goal of greater understanding was still achieved. This participant in particular was delighted with the system, expressing the need for and potential of it. During the evaluation process many participants were found to ask questions and relate their learning to their own experiences.

Participant 8 whom had the vignettes correct initially so had no room for improvement, specified: ``I thought this was a great tool and it took me 4 tries to complete the mission. I learnt new things (about fluorescent lights and lights in general).��

It was interesting to observe the different strategies being used by participants to solve the problems. Only one person realised they could look away from the lights in order to prevent a sensory overload (there was no hint for this given). Some people would try to go back to the livingroom or bedroom during the routine in order to refill contentment before continuing. It is uncertain as to whether this should be allowed, because it is not directly part of the strict routine, but it does convey why children with autism need to be able to use their special interests to deal with the surroundings. Consultations with people with autism would need to be conducted. It could be that the user has a rough routine in the form of goals i.e ``Go to kitchen, Brush teeth, Get dressed��, but they have to develop their own routine to achieve these goals. So, if they get dressed first and their contentment lowers, they now have to complete the rest of the routine with a fixed lower contentment and then may find achieving the rest of the goals more difficult.

Overall, it seems that this would be a better way of teaching people about why children with autism require routines. In the formative evaluation, most people commented more on the sensory overloads aspects rather than the need for routine. However, following later improvement in the simulator and greater focus on learning, it seems that of the latter 4 participants, 3 did pick up on the need for routine, so it may have been due earlier to previous problems of the environment being too fast paced. However, it is uncertain if ``why�� a routine was needed was conveyed, rather than simply the fact that ``it's just needed to deal with sensory issues��; effectively it is hard to gauge if the candidates acquired declarative knowledge on the topic rather than procedural knowledge.  

\subsection*{Problems}

The obvious problem with this approach was that it allowed for vignettes not to be completed equally. Some people filled out one vignette and not the other, so some data could not be used. However this was expected and could have been better designed for by an automated system of emailing people the link to the first vignette and a link to download the simulator and after completion of simulator sending an email from the simulator to them with a link to the second vignette. Although possible, it was decided not to do this and vignettes have been completed equally. They could also have been required to select 3 items in each.

The second identified problem could be that teachers could view both vignettes and decide for themselves which one was easiest and complete this first. This could work in favour of the evaluation if they chose the vignette they were certain of and got this correct, and were able to complete the second vignette they may previously have been uncertain of and get this correct. This means that understanding could be increased but the results would not indicate this. However what should be seen is that teacher understanding is not made worse. 

In spite of these problems, the online evaluation was the first opportunity to see if people could use the simulator, and complete the evaluation, without being able to ask questions, either on game play or if they were stuck. None of the teachers whom completed the online evaluation emailed to ask what they needed to do or asked for help, although some emailed that they had issues running the simulator: this was determined to be due to exceptionally old computers. The minimum computer specification required needs to be identified in future work.  

For in person evaluations, most of the improvements were related to those who did Vignette A first and the participants with biggest responses/improvements both did A first. This could indicate that vignette B. 

%% text missing here?

There also may have been problems with the scoring of the vignettes. For participant 5 of the on-line survey, their responses were ``Find Johnny and ask him what is wrong, Offer Johnny a chance to work on his own, Go outside and ask the gardener to stop mowing the lawn.�� Thus although they received the highest score, they omitted ``closing the window, which is related to a sensory response. This was given the same weight as decisions not related to sensory responses. In the broader picture their response is probably a good one, however the evaluations goals were designed to look specifically at sensory problems, so non-sensory response should possibly have received a lower initial score. Having one sensory related option in B in comparison to two in A could indicate why vignette might be easier, however such problems were not found until all of the data was collected. 

The online version lost data due to some people who completed the first vignette and background questionnaire not completing the second. This is unfortunate because quite a few of these participants gave initial low scoring answers and thus there may have been seen to be a potentially large improvement. It is also noted that a few teachers with initial low scoring answers expressed that they felt that they had good knowledge of autism and were confident of managing behaviours, in spite of having not had training. In contrast, those whom seemed to have large amounts of experience of working with autism were expressing less knowledge and confidence in dealing with behaviours (when the opposite might be expected). 

Finally background results could be skewed as two participants realised they were inverting their responses and needed to redo them. There may possibly have been others who also did this and did not notice. However general background responses seem to fit with the courses taken. This was changed in the on-line version so that participants checked boxes for "Strongly agree", etc, rather than having to assign numerical values to responses: the potential for background data not being correct is however not deemed strong enough to affect conclusion of results. 

\section{Conclusions}

Improvement was generally seen in participants. A large portion of the participants came with a lot of prior knowledge of autism. In spite of one vignette potentially being easier than the other, it can be concluded that the simulator did not hurt participants understanding of autism, and may have increased understanding in spite of there being no significant increase. 
%%% as was demonstrated in the case of participant 8 � not clear how this 
%%% supports this
Some feedback provided after using the simulator also supports this. For the one participant whom had an overall reduction in score, they did appear to show improvement of sensory understanding items, so the goal may still have been achieved, to some extent. 
%% weakened the claim here

Overall the results are positive, although there 
%% are not potentially enough participants 
is not sufficient evidence to reach a firm conclusion. It was extremely useful to see people using different methods of thinking and strategies and it is felt this could be a good platform for discussion of strategies. Some people were very determined to complete it; some people had a little difficulty, but were still very positive about it. However, difficulty of completing the tasks and missions still needs to be reduced: the goal is to inspire and motivate the user to think of their approaches and potential strategies, but not be so difficult that users become frustrated or bored. 

\begin{quote}
``Overall this is a genius idea, really useful and I think it could be really good for those working with ASD children and adults (you could also have a child and adult version - are there differences?)�� - participant 8
\end{quote}

\subsection{Suggested future improvements}
Suggested future improvements from the summative evaluations:
\begin{enumerate}
\item Allow clicking of the mouse and not just the space bar.
\item Have controls displayed on the screen to reduce cognitive load for new users needing to memorise what the controls are. 
\item More strategies and hints to be given. 
\item More needs to be done to address difficulty.
\end{enumerate}


