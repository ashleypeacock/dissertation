\documentclass[11pt]{report}
\usepackage{fullpage}
\usepackage{float}
\restylefloat{table}
\begin{document}

\title{Autism Simulator}
\author{Ashley Peacock}
\maketitle
\tableofcontents
\chapter{Introduction}

\section{Selecting a project}
The project started with the purpose of creating software to benefit someone with autism or ADHD or those in contact with these conditions such as family members or carers. Owing this is a very broad topic it was important to create multiple proposals and select the most useful. All proposals were put on a website and a selection was made after considering results from an online survey, conversing with professionals and people with ASD and considering systems and research currently available.

Project proposals:
\begin{table}[H]
    \begin{tabular}{| p{4cm} | p{9cm} |}
    \hline
    Proposal name & Description                                                                  \\
    \hline
    \hline
    Online diary & Online system to improve communication between carers, parents, social workers, schools. Parties could post questions and ask for suggestions when dealing with certain behaviours as well as document the child's day allowing easier identification of patterns of behaviour or problems                    \\
    \hline
    Social simulator & Simulated social scenarios for autistic users to trial various social situations and see possible outcomes  \\
    \hline
    Dynamic scheduler and planner app & A planner that would re-schedule tasks when not completed and present basic to-do lists with tasks broken down into manageable chunks  \\
    \hline
    Environment app & Phone app aimed to encourage children to look and question their environment \\
    \hline
   Autism simulator & A 3D virtual environment where the user plays as a child with autism and can thus experience some of the obstacles faced through a visual/game environment \\
    \hline
    \end{tabular}
\end{table}

\subsubsection{Questionnaire}
A questionnaire was anonymously completed by six people in total and included people with ASD/ADHD, professionals, carers and parents and was compiled with both qualitative and quantitative questions.

\begin{enumerate}
\item Please give some information about yourself, for example if you have ASD/ADHD or are a professional/carer.
\item Please select and rank three proposals you feel are the best
\item Please explain reasons for selection
\end{enumerate}

\subsubsection{Results}

Below summarises some of the comments given in the feedback questionnaire as well as considerating factors from other areas

\begin{table}[H]
    \begin{tabular}{| p{4cm} | p{8cm} |}
    \hline
    Proposal name & General reasons for/against                                                                 \\
    \hline
    \hline
    Online diary & For: Cross communication between doctors, teachers, parents and carers which is often problematic with information missed. Against: Good in theory but may not be practical due to data protection. Relies too heavily on parents/carers being able to read emails or notifications. May be difficult for some schools to gain access to wifi.                \\
    \hline
    Social simulator & Against: Big project given the time-frame. Other companies working on a similar concept. Much research on this topic already. Conveying 'social stories' could be a better approach to deal with context specific situations. \\
    \hline
    Dynamic scheduler and planner app & Against: least unique proposal, many other planners available. For: No planners available that specifically target planning/executive functioning difficulties within ADHD and Autism \\
    \hline
    Environment app & Against: Hard to back with literature. Difficult concept to understand(possibly not explained well) For: Least amount of implementation work. Could be simply but effective. \\
    \hline
   Autism simulator & Against: Big project given the time frame, no previous simulators which can be drawn from. For: Most unique and popular idea. Misunderstanding from the general public is a big problem. Could be extremely helpful for teacher's training. \\
    \hline
    \end{tabular}
\end{table}

After considering feedback from the questionnaire, further feedback from the autism community, plausibility given the time constraints, usefulness to the community, originality and current skill-set the 'Autism simulator project was selected.

\chapter{Literature review}

\section{What is autism}

Autistic Spectrum Disorder is a lifelong condition which affects how an individual communicates and may perceive the world around them \cite{nas}. It is currently diagnosed by the presence of atypicalities in three domains(collectively known as the triad of impairments): social imagination, social communication, social interaction. In addition to these are non-diagnostic but highly prevalent features such as sensory abnormalities, information processing difficulties and prosopagnosia. 

With some of the disadvantages that may come with having autism, there are reported strengths as a result of having a unique cognitive style, for example a talent for spotting details\cite{bayes} or having a good long-term memory for encyclopaedic knowledge of their 'special interests'. As a spectrum disorder the severity and type of difficulties those with autism have, vastly differ and are unique to each individual.


\subsection{Triad of Impairments}

There are three key areas of difficulty that people with autism share.

\subsubsection*{Social communication}
People with autism have difficulty with verbal and non-verbal language such as body language or tone of voice. Language tends to be interpreted literally and thus metaphors, sarcasm and jokes can be difficult to understand\cite{nas}. An example of literal interpretation is where a person with autism misunderstood the question "What's up?" and proceeded to look up at the ceiling. Other communication difficulties include echolalic language(repeating language said to them) or speaking excessively about their 'Special interests' without detecting that the other party may be bored\cite{nas}. Although people with autism will usually understand what is being said to them they may prefer to use visual symbols such as PECS(Picture Exchange Communication System). 

Due to literal language interpretation, it is important that language communicated is clear, concise and unambiguous, one of the needs the public were most unaware of\cite{autismmisconception}.

\begin{quote}
Most things I take at face value, without judgement or interpreting them. I look at them in a concrete, literal and very individual way. \cite{olgab}
\end{quote}

\subsubsection*{Social interaction}

\begin{quote}
Autistic people have to understand scientifically what non-autistic people already understand instinctively 
- Mark Segar, Autistic Survival Guide.
\end{quote}

Many people with autism have difficulty giving eye contact, one person described eye contact as "physically painful". By not giving eye contact, it may cause social queues such as facial expressions to be missed, potentially leading to inappropriate responses. Lack of eye contact could be perceived as rude or not paying attention to the speaker, causing possible unintentional offence. Other social interaction difficulties reported include trouble understanding social rules\cite{nas}, for example why people say 'thankyou'.

\subsubsection*{Social imagination}
Social imagination deficits result in difficulties 'Putting themselves in another person's shoes', also known as 'Theory of mind'. Other resulting difficulties include problems predicting events or identifying possible dangers such as running across a road and consequently, new situations can be difficult.\cite{nas}

Social imagination difficulties can make it hard for a child with autism to engage in imaginative play, preferring to act out scenes from films identically which can make it difficult for other children interacting with them if they prefer to deviate or explore a new plot.\cite{nas}

\subsection{Information processing}

It is suggested that people with autism process information holistically, a theory known as Gestalt perception. Gestalt perception is posited to be a cause of fragmented or distorted perceptions in people with autism\cite{olgab}; processing information as a whole instead of in parts make it difficult to drawn connections and thus make predictions about the world.  "I had always known that the world was fragmented. My mother was a small and a texture, my father was a tone, and my older brother was something, which was moving about" \cite{williams1992}. 

It is argued that people with autism perceive the world more accurately because their inferences are less dependent on previous experience but a negative consequence of this is being less able to filter irrelevant stimulus \cite{bayes}. Difficulties filtering information can cause problems differentiating between background and foreground noise and so in a room with many people talking it may be hard to tune into an individual conversation \cite{bayes}. 

Delays in information processing are a common feature in autism. In extreme cases, it can take weeks, months or even years to process information and one of the reasons given to the cause lye in the theory of gestalt perception. Processing information as a whole leads to over-selectivity and thus even familiar environments are looked upon as entirely new and thus one small change to the environment can cause a large amount of distress\cite{olgab}. This would offer a suggestion as to why people with autism have a strong desire for strict routine. Questioned asked to a person with autism should be given ample time for a response, if their process of thought is interrupted it can cause a complete disruption and the individual has to start this process again\cite{olgab}. As a result of distorted perception, it may take someone with autism longer to adjust to their surroundings. 

Distortions are reported to become worse in the state of nervous over-arousal and information overloads\cite{olgab} and thus a cycle of problems occurs; the more stressed a person with autism may be, the more these distortions occur and the harder it is to make sense of the world, consequently resulting in even more stress.

\subsubsection{Sensory processing}

While social and communication difficulties are core symptoms and most commonly associated with autism in the public view, "Many people with Asperger syndrome/High functioning autism define their sensory processing problems as more disabling than the deficits in communication/social behaviour\cite{olgab}. Sensory processing differences in autism are highly reported, 81\% of respondents reported differences in visual perception, 87\% in hearing, 77\% in tactile perception, 30\% in taste and 56\% in smell \cite{sensory_leisure}. Senses play a vital role in how we model and perceive the world around us so if one senses the world in a differently, their view and resulting behaviours will also be different. 

Senses in autism can be hyper(more sensitive), hypo(less sensitive), agnostic or fluctuate between hyper and hypo\cite{bayes}. As with all areas of autism, sensory atypicalities differ and are unique to each individual, however, these fluctuations make it an area of particular challenge for carers and for a person with autism to identify or predict troubling sources before they occur. Fluctuations can be described as a 'FM radio that is not exactly tuned on the station when you are driving down the freeway. Sometimes the world comes in clearly and at other times it does not" \cite{olgab}.

When a sensory channel is in a state of agnosia, although able to see, one may not be able to assign it to any meaning. The result is one can become 'mind-blind', or 'mind-deaf' where the person can appear as if they are genuinely deaf.

Catering with for the many different sensory needs for many different children can be very demanding. In the classroom if a child is hypo-visual and feels a need to stimulate their visual senses by constantly switching on and off a light, in contrast to another child in the class being hyper sensitive, the result could lead to a sensory or information overload(this was commented on in one of the interviews from the teacher...).

Below are some examples of the effects someone with autism may experience depending on the state of their sensory channel:

\begin{table}
    \begin{tabular}{| l | p{5cm} | p{5cm} |}
    \hline
    Sense channel & Hyper                                                                                                                      & Hypo                                                                   \\
    \hline
    \hline
    Vision        & Vision may be magnified                                                                                                    & Attracted to light or fascinated with bright colors                    \\
    \hline
    Auditory      & Sounds are amplified. Temple Grandin a write with autism described her ears as like 'microphones'                          & Is attracted to sounds/noises                                          \\
    \hline
    Tactile       & Clothes may hurt. One person with autism described clothe labels as feeling like 'barbed wire'. May not like being hugged. & Enjoys being hugged or seeks pressure by crawling under heavy objects. \\
    \hline
    Taste/Smells & Smells or texture of foods may be intolerable. & Mouths and licks objects \\
    \hline
    Vestibular & Difficulty with walking or crawling on uneven or unstable surfaces. & Spins, runs round and round, rocks back and forth \\
    \hline
    \end{tabular}
\end{table}

// (below is probably not much use at the moment, but useful for later justifying the game character's traits and responses to the environment if I can structure it in properly...)

Sensory processing patterns can be categorised into four-types\cite{sensory_leisure}:

\begin{enumerate}
\item Sensory avoidance pattern: Low sensory threshold. 
\item Sensory seeking patterns: A high sensory threshold and make seek out stimulus.
\item Sensory sensitivity patterns: Low thresholds and may respond to stimulus more intensely or for a longer period of time.
\item Low registration: High sensory threshold, may appear not to detect incoming sensory information and also show a lack of responsiveness.
\end{enumerate}

Correlation between sensory difficulties and difficult temperament characteristics such as activity level, adaptability to changing context, quality of mood, threshold of responsiveness, intensity of reaction and persistence\cite{temperament}. 

\subsubsection{Sensory and Information overload}

When the amount of information required to be processed comes in large volumes and too quickly to processes it can result in someone with autism experiencing an 'Information' or 'Sensory' overload. Overloads can result in hypersensitivity causing lights becoming brighter or sounds becoming louder. Visual/auditory causes of overloads can cause tactile sensitivity and so being touched might be painful whilst experiencing a sensory overload could possibly be painful. Donna Williams reports that "sensory overload caused by bright lights, fluorescent lights, colours, and patterns makes the body react as if being attacked or bombarded, resulting in such physical symptoms as headaches, anxiety, panic attacks or aggression"\cite{bayes}.

The resulting behaviours again differ for each individual and are discussed in the following section.

\subsection{Resulting behaviours}


\subsubsection{Meltdowns}
If a sensory overload is not dispersed quickly enough it can lead to a full sensory shut-down in which all senses enter a state of agnosia and the person with autism withdraws from the world. Another reaction to a sensory overload is entering a state of 'fight or flight', running away from the source without any sense of danger, or exhibiting temper-like tantrums or self injurious behaviour. These behaviours can be collectively known as 'meltdowns'; the individual experiencing them feels a loss of control. Meltdowns can be caused by not only by sensory, but an emotional and cognitive overload.

\subsubsection{Mono-processing}
Mono-processing is described as an involuntary response to information overloads where all but a few sensory channels are closed. Vision may become hyper-sensitive whilst but the individual may not be able consciously hear. Subconsciously however, this information may be absorbed and processed later, further increasing the information processing delay. 

\subsubsection{Unusual fears}
It was found that 40\% of children with autism had unusual fears in comparison to 0-5\% of typical children, the vast majority of these were made up of mechanical objects. Children with autism have higher levels of anxiety than typical children\cite{fears} and increased anxiety from being faced with more fears on a day to day basis will only increase this and further impact on functioning. For example, not leaving the house because it's cloudy, or not taking a shower because of the noise from the drain, not going to school due to being afraid the fire alarm will sound. The top five reported unusual fears were toilets, elevators, vacuum cleaners, thunderstorms, tornadoes. The cause of many of these unusual fears in children with autism are thought to be related to sensory perception differences\cite{fears}.

\subsubsection{Repetitive and restricted behaviours}

// Note: find information on 'attractive stimulus' and Sensory soothing objects. Relate it more to content that can be use as justification for simulator choices.

Repetitive and restrictive behaviours are highly prevalent in people with autism and are thought to be caused by:
\begin{enumerate}
\item Needing to induce sensory sensory stimulation\cite{rrsyouth}.
\item As a reaction to sensory stimulation\cite{rrsyouth}.
\end{enumerate}

Repetitive behaviours and sensory issues have been found to be positive correlated\cite{rrs_sensory}\cite{rssyouth}. High levels of restricted behaviours were associated with less severe levels of depression, indicating that such behaviours may act as a mechanism to protect against or be a direct cause\cite{rss_ensory}. Those with low-functioning autism were more likely to engage in repetitive behaviours such as 'stimming', repetitive manipulation of objects and self-injurious behaviour in contrast to high-functioning autism having restricted interests, language or attachment to objects\cite{rss_sensory}. People with high-functioning autism were reported to have higher levels of anxiety with restrictive behaviours thought to a developed coping mechanism\cite{rssyouth}.

93\% of children with autism were reported to be distressed by change \cite{fears}. With an ever changing perceptions of the environment, routine can be their only sense of familiarity and reassurance. Interestingly it is reported that people with autism can have more problems with small changes in a familiar environment in comparison to entirely new situations\cite{bayes}.  


\section{Impact and Prevalence}
Figures drawn from the 2011 census estimate that 1.1\% of the population have Autism\cite{nas}. This figure appears to be rising across the globe as awareness and understanding of the condition increases alongside broadening criteria\cite{increasingprevalence}, Aspergers syndrome is one example addition and which has only been a formal diagnosis since 1990. Early counts of people with autism spectrum conditions were less than 10 in 10,000, this has grown to a new prediction of 110 in 10,000 in the USA \cite{increasingprevalence}.

It is estimated that only 22\% of teachers have been trained specifically in autism and the majority of training given is typically one to four hours. 54\% of all teachers in England do not feel they have had adequate training to teach children with autism.\cite{statsandfacts} 30\% of parents of children with autism in mainstream education are satisfied with the level of understanding of autism across the school\cite{nasschool}. 23\% of parents are dissatisfied with SENCO's level of understanding of autism. 

Figures obtained show that approximately 40\% of children with autism have been bullied at school. 1 in 5 children with autism have been excluded from school \cite{nasschool} and only 24.4\% of pupils with autism achieved 5A*-C GCSEs in 2010/2011 in comparison to 58.2\% of the overall population\cite{statsandfacts}, a surprising figure owing that people with autism are deemed to have average to above average intelligence which indicates difficulties at school may be a reason for not reaching their potential.

\begin{quote}
Danny would not have been excluded if the school had understood the difference between 'normal' behaviour and Aspergers syndrome. They inflamed situations because they didn't understand that my son finds physical contact, or being touched by teachers, really difficult \cite{nasschool}
\end{quote}

\begin{quote}
If I could make one change...I would ensure compulsory, thorough training about autism and how it affects learning is given to all school staff. \cite{nasschool}
\end{quote}

\section{Public perception of autism}
Although there is some level of awareness of autism in the public domain, there is still much left to be desired.  From a survey carried out by the National Autistic Society, 92\% of respondents had heard of Autism but only 48\% had heard of Aspergers syndrome which has less obvious difficulties and is consequently regarded as a 'hidden disability'. Most were able to identify key characteristics of autism such as difficulty communicating or making friends. Other common characteristics such as a need for 'clear unambiguous instructions' and sensory hypersensitivity were less known\cite{autismmisconception}. 

// find sources on public interaction with people who have autism.

\begin{quote}
If I could make one change... every person who comes into contact with my daughter would have some form of training in autism.\cite{nasschool}
\end{quote}


\section{Previous work}

\subsection{Education software}
How other education software can be used to help children with understanding autism or general learning. Include information from the paper Alyssa sent.

\subsection{Other simulators}
Include the one that was released this year and possibly other simulators that have been used to convey other disabilities if I can find them...

Perhaps offering an argument as to why simulators may not be a good idea/useful.

\section{Other}
Just for now, some information that might be of use to put somewhere, not sure where.

\begin{quote}
Action to increase understanding of autism across the whole school and to provide support with social activities can make a huge difference to whether a child with autism feels included at school.\cite{nasschool}
\end{quote}

"if you deal with 'challenging behaviours' in autism, do not focus on the iceberg; do understand the underlying causes of the behaviours and try to develop an approach not based on symptoms but on prevention. Challenging behaviours are caused by problems of communication, social understanding, by different imagination, by sensory problems...Therefore try to understand autism 'from within'. It is easier said than done, because it requires an enormous effort of imagination: we need t learn to put ourselves in the brains of autistic people and then we will understand better through their eyes the obstacles in their attempts to survive among us" - Theo peeters \cite{olgab}

\begin{quote}
It doesn't appear that mainstream teachers have had access to training. The fundamental issues relating to communication, behaviour and language disorder continue to be misinterpreted as 'bad behaviour', 'not listening' and so on.\cite{nasschool}
\end{quote}



\chapter{Design process}
As the project selected has a very large scope it was important to identify the most important goals and decide on restrictions. Autism as previously described comes with a vast amount of difficulties, some of which may be too complex or time consuming to convey(such as social difficulties). Owing the vast environments a child can be exposed to on a day to day basis(school, work, parks, etc), a house was chosen as this is the place we will most often be and with understanding the pitfalls and hazards around the house for a person with autism, understanding could then be generalised for the player to other environments. Once the environment was selected, interviews and a consultation from the LAER group aided narrowing which autism difficulties may be the most important.

\section{Interviews}

Interviews were carried out with five people from varying backgrounds and exposure to autism:

\begin{enumerate}
\item Candidate one: teacher of a school for autistic children
\item Candidate two: special needs teacher of a school with varying disabilities.
\item Candidate three: parent of a teenager with Aspergers syndrome and ADD. Described themselves as neurodiverse having severe sensory difficulties but less social ones.
\item Candidate four: parent of a child with Aspergers syndrome and is themselves neurodiverse. Candidate describes having high sensory issues and less social ones.
\item Candidate five: person with high-functioning autism whom has higher social difficulties and less sensory.
\end{enumerate}

\section{Difficulties chosen}
Following interviews and reviewing literature available, the following aspects of autism selected are:

\begin{enumerate}
\item Sensory atypicalities: selected as the primary difficulty to convey due to their prevalence and hidden nature which is less known to the public
\item Meltdowns: As these can be caused by sensory atypicalities and it is important to convey to the user the impact of difficulties, not just the difficulties themselves.
\item Special interests: A means in the game to 'soothe' the character and counteract meltdowns.
\item Ambiguous instructions and processing delays: commented as a problem in the classroom.
\end{enumerate}

\section{Game design}

\subsection{Autism aspects}

What aspects of autism and discussion/ideas of how they will be conveyed and why it will be conveyed in this way/what the visualisation will represent.
What was chosen and why it was chosen.
Mock ups.

\subsection{Design of sensory system}

\subsection{Character}
The character the user will play as. What difficulties they have/ what age they are.

\subsection{Story boards}

\subsection{Tool selection}
For the first version of the simulator a game engine will be used, allowing focus to be directed on the higher level aspects and quicker development. Blender will be used as the modelling tool as it is freely available, powerful and well supported with lots of tutorials and documentation.

\subsubsection{Game engines}

\textbf{Unity}\\
Unity is one of the most popular game engines available with good support for models. Unfortunately the licence costs 1500 and the free version comes with limitations.

Advantages: popular game engine to use. Quick development with scripting. Phone app support.
Disadvantages: Interface heavy, steep learning curve, limited to just scripting, costs, good computer required to run it efficiently.

\textbf{JMonkey}\\
JMonkey is a java 3d game engine that has been in development around for a few years. It has an extremely active and helpful community, allows complete customisation and holds little limitation being open source.

Advantages: Provides development environment with scene graph. Active community where you often get responses from developers themselves. Java is quick to develop in. Support for online use and phone apps.
Disadvantages: Java is not seen as the preferred language for graphics or games.

\textbf{Panda3D}
Originally created by Disney, Panda3D is an engine which can be used via python or C++ although support is mostly for python.

Advantages: Quick to develop for with a choice in language. Good community with lots of tools.
Disadvantages: No phone app and limited online support. Lack of documentation. 

\textbf{Ogre3D}
Ogre3d is primarily a graphics rendering engine and but it does have additional plugins such as 'physics' or drawing interfaces.

Advantages: Lots of modules and plugins. Powerful and used commercially. Active support community.
Disadvantages: Longer development process. Lack of tools such as a scene graph. No support for putting online.

Jmonkey was chosen due to it's active community, speed of development/development environment and ease to put online. An ability to put the project online will increase availability. Although C++ would be much faster for the user, JMonkey doesn't have a reduction in performance that counteracts it's speedier development. 


\chapter{Prototype}

\section{Implementation}
What was in the prototype. Overall concept and pictures of house.

\subsection{Technical}
- Explanation of missions interface
- Scenes
- object states
- player class
- gui
- action manager
- Class diagrams

\subsection{Autism aspects}
Which aspects of autism were conveyed and what the final results looked like.

\subsection{Sensory system}
- meltdowns
- contentment
- sensory level
- filters

\subsection{Interface}
\section{Evaluation}
\subsection{Expert feedback}
\subsection{User feedback}
\section{Improvements planned}

\chapter{First version}

\section{Storyboards}

\section{System changes}

\subsection{Overview}

Followed by more technical sections on:
- rewrite of scenes.
- introduction of gamestate manager.
- rewrite of sensory system.
- GUI changes

\section{House design}
House design choices. 
Put list of objects here and a house plan. 

\chapter{Formative evaluation}

\chapter{Final version}

\chapter{Summative evaluation}

\chapter{Release}
- Website
- Provided materials
- Response

\chapter{Conclusion}
- Discussion, i.e things that could have been done better.
- Future improvements

\begin{thebibliography}{9}

\bibitem{increasingprevalence}
Johnny L. Matson, Alison M. Kozlowski
The increasing prevalence of autism spectrum disorders. Research in Autism Spectrum Disorders(2011)

\bibitem{nas}
National autistic society. www.autism.org.uk

\bibitem{statsandfacts}
Ambitious about autism. www.ambitiousaboutautism

\bibitem{nasschool}
Make school make sense. Autism and education: the reality for families today. National Autistic Society, 

\bibitem{autismmisconception}
Autism misconceptions. NHS

\bibitem{olgab}
Sensory Perceptual Issues in Autism and Aspergers syndrome(2003). Bogdeshina, O

\bibitem{fears}
Unusual fears in children with autism(2012). Susan Dickerson Mayes, Susan L. Calhound, Richa Aggarwal, Courtner Baker, Santosh Mathapati, Sarah Molitoris, Rebeccas D. Mayes.

\bibitem{temperament}
Sensory correlates of difficult temperament characteristics in preschool children with autism(2012). I-Ching Chaung, Mei-Hui Tseng, Lu Lu, Jeng-Yi Shieh

\bibitem{bayes}
When the world becomes 'too real': a Bayesian explanation of autistic perception(2012). Elizabeth Pellicano and David Blurr.

\bibitem{williams1996}
Autism: An inside-out approach(1996). Williams, D

\bibitem{williams1994}
Somebody Somewhere: Breaking Free from the World of Autism(1994). Williams, D.

\bibitem{williams1992}
Nobody no-where(1992). Williams, D

\bibitem{sensory_leisure}
Sensory processing abilities and their relation to participation in leisure activities among children with high-functioning autism spectrum disorder. Hochhauser, M. Engel-Yeger, B.

\bibitem{sensory_toddlers}
Parent Reports of Sensory Symptoms in Toddlers with Autism and Those with Other Developmental Disorders. Sally J. Rogers, Susan Hepburn, Elizabeth Wehner

\bibitem{sensory_children}
Describing the sensory abnormalities of children and adults with autism(2007). Susan R. Leekam. Carmen Nieto. Sarah J. Libby. Lorna wing. Judith gould. 

\bibitem{sensory_perceptual}
Sensory integration and the perceptual experience of persons with autism(2006). Grace Iarocci. John McDonald.

\bibitem{rss_cognitive}
Restricted and Repetitive Behaviours, Sensory Processing and Cognitive Style in Children with Autism Spectrum Disorders(2009). Yu-Han Chen, Jacqui Rodgers, Helen McConachie.

\bibitem{rrsyouth}
Restricted and repetitive behaviours and psychiatric symptoms in youth with autism spectrum disorders(2013) Elizabeth A. Stratic, Luc Lecavlier.

\bibitem{rrs_sensory}
Is there a relationship between restricted, repetitive, sterotyped behaviours and interests and abnormal sensory response in children with autism spectrum disorders?(2008). Robin L. Gabriels, John A. Agnew, Lucy Jane Miller, Jane Gralla, Juliet P. Dinkins, Elizabeth Hooks.


\end{thebibliography}

\end{document}