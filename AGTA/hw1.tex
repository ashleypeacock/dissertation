\documentclass[11pt]{article}

\newcommand{\define}[2] {
  \textbf{Definition: #1}
  \begin{center} #2
\end{center}
}

\begin{document}

\title{AGTA HW1 \\ S0936300}

\maketitle

\section{Question 1}

Given the matrix: 

\begin{center}
$ \bordermatrix{~ & ~    & ~    & ~ & ~    \cr 
				~ & (6, 5) & (4, 8) & (6, 4) & (9, 2) \cr 
				~ & (4, 6) & (7, 4) & (7, 5) & (4, 4) \cr
				~ & (4, 7) & (4, 4) & (9, 5) & (2, 6) \cr
				~ & (5, 9) & (4, 10)& (4, 9) & (8, 9) \cr}$ 
\end{center}
				
To compute Nash equilibrium we start by removing dominated strategies. 

\begin{enumerate}
\item Column 4 is weakly dominated by Column 1.
\item Column 3 is weakly dominated by Column 1.
\item Row 4 is is weakly dominated by Row 1.
\end{enumerate}

Removing the dominated strategies we are left with:

\begin{center}
$\bordermatrix{~ & ~      & ~  \cr
			   ~ & (6, 5) & (4, 8) \cr
			   ~ & (4, 6) & (7, 4) \cr
			   ~ & (4, 7) & (4, 4) \cr} $
\end{center}
			   
Row 3 is dominated by row 1 and 2. This leaves us with the following game:

\begin{center}
$\bordermatrix{~ & A      & B  \cr
			   C & (6, 5) & (4, 8) \cr
			   D & (4, 6) & (7, 4) \cr} $
\end{center}

There is no pure strategy Nash Equilibrium. Nash's theorem however states that every finite game does have at least one mixed nash equilibrium.

Let $q$ be the probability that player 2 plays strategy A and $(1 - q)$ the probability player 2 plays strategy B.
Let $p$ be the probability that player 1 plays strategy C and $(1 - p)$ the probability player 1 plays strategy D.

For player 1 to be indifferent to player 2:
\begin{center}
$6q + 4(1 - q) = 4q + 7(1 - q)$ \\
$13q = 8q + 3$\\
$q = 3/5$\\
\end{center}

For player 2 to be indifferent to player 1:
\begin{center}
$5p + 6(1 - p) = 8p + 4(1 - p)$\\
$-p + 6 = 4p + 4$\\
$5p = 2$\\
$p = 2 / 5$
\end{center}

Profiles:\\
	$x_{1} = (2/5, 3/5, 0, 0)$ \\
	$x_{2} = (3/5, 2/5, 0, 0)$ \\
	
We can see that both of these profiles are nash equilibrium as by choosing these profiles, the players become indifferent to one another. There are no other pure strategy equilibrium however, by using weak dominance there is a possibility of other mixed strategy equilibriums that were removed. 

\section{Question 2}

Linear problems used:
\begin{center}
$5a + 7b + 2c + 4d + 3e >= v$ \\
$2a + 3b + 4c + 9d + 6e >= v$ \\
$a + 5b + 8c + 3d + 6e >= v$ \\
$9a + 6b + c + 4d + 4e >= v$ \\
$5a + 2b + 7c + 6d + 2e >= v$ \\
$a + b + c + d + e = 1$ \\
$a >= 0, b>=0, c>=0, d>=0, e>=0$
\end{center}

Where $v$ is the value we are trying to maximise.

Result:\\
\begin{center}
$a = \frac{48}{335}, b = \frac{23}{67}, c=\frac{82}{335}, d = \frac{17}{67}, e = \frac{1}{67}$
\end{center}

Value of game: 
\begin{center}
$v = \frac{1564}{335}$
\end{center}

\section{Question 3}

\subsection{Part A}
Matching pennies has a unique mixed strategy Nash equilibrium. Both players have the strategy profile $x = (\frac{1}{2}, \frac{1}{2})$

\subsection{Part B}

This part was proven experimentally. The experiment was carried out with a different number of trials and it could be observed that with an increasing amount of trials/games played that the probabilities further converged. No noticeable differences were observed when choosing different starting strategies and all four possible starting strategies were experimented with. No noticeable differences were observed when choosing a different tie break.

In the case of a tie break, tails was chosen. Code is attached.

(unsure of proving mathematically, could use summations but was unsure if this was the correct approach)

\subsubsection*{Results}
Please see appendix for an example of a full, step by step print out of a game converging and resulting probabilities when played over a number of trials.

We can see from the results that the final probabilities tend to converge to 0.5. We can also see in the results where the probability is printed after each move that the results tend to 0.5.



\section{Question 4}

\subsection{Part B}

Solution to LP is(using maple)
\begin{center}
$x = \frac{78}{47}, y = \frac{79}{47}, z = \frac{12}{47}$
\end{center}

\textbf{Dual is:} \\
$Minimize: 5x_{1} + 6x_{2} + 8x_{3}$\\
\begin{center}
\textbf{Subject to:} \\
$2x_{1} + 3x_{2} >= 2$\\
$x_{1} + 4x_{3} >= 3$\\
$4x_{2} + 5x_{3} >= 5$ \\
$x_{1}, x_{2}, x_{3} >= 0$
\end{center}

Solving using maple gives:

\begin{center}
$ y = \frac{20}{47}, \frac{17}{47}, \frac{31}{47}$
\end{center}

\section{Appendix}

\subsection{Question 3: Full print out of a trial}
Below is a full print out of the probabilities at each step after a move has been played. This was played over a series of 15 moves and we can see that a tie break is often reached, showing that the probabilities often converge to 0.5 before changing because of the tie break. Starting strategy was (1, 0) (player 1 plays heads, player 2 plays tails) \\

P2 will play H with probability 0.0  P1 plays tails \\
P1 will play H with probability 1.0  P2 plays tails\\
P2 will play H with probability 0.0  P1 plays tails\\
P1 will play H with probability 0.5  we're in a tie break\\
P2 will play H with probability 0.0  P1 plays tails\\
P1 will play H with probability 0.333333333333  P2 plays heads\\
P2 will play H with probability 0.25  P1 plays tails\\
P1 will play H with probability 0.25  P2 plays heads\\
P2 will play H with probability 0.4  P1 plays tails\\
P1 will play H with probability 0.2  P2 plays heads\\
P2 will play H with probability 0.5  we're in a tie break\\
P1 will play H with probability 0.166666666667  P2 plays heads\\
P2 will play H with probability 0.571428571429  P1 plays heads\\
P1 will play H with probability 0.142857142857  P2 plays heads\\
P2 will play H with probability 0.625  P1 plays heads\\
P1 will play H with probability 0.25  P2 plays heads\\
P2 will play H with probability 0.666666666667  P1 plays heads\\
P1 will play H with probability 0.333333333333  P2 plays heads\\
P2 will play H with probability 0.7  P1 plays heads\\
P1 will play H with probability 0.4  P2 plays heads\\
P2 will play H with probability 0.727272727273  P1 plays heads\\
P1 will play H with probability 0.454545454545  P2 plays heads\\
P2 will play H with probability 0.75  P1 plays heads\\
P1 will play H with probability 0.5  we're in a tie break\\
P2 will play H with probability 0.692307692308  P1 plays heads\\
P1 will play H with probability 0.538461538462  P2 plays tails\\
P2 will play H with probability 0.642857142857  P1 plays heads\\
P1 will play H with probability 0.571428571429  P2 plays tails\\
P2 will play H with probability 0.6  P1 plays heads\\
P1 will play H with probability 0.6  P2 plays tails\\
P2 will play H with probability 0.5625  P1 plays heads\\
P1 will play H with probability 0.625  P2 plays tails\\
P2 will play H with probability 0.529411764706  P1 plays heads\\
P1 will play H with probability 0.647058823529  P2 plays tails\\
P2 will play H with probability 0.5  we're in a tie break\\
P1 will play H with probability 0.666666666667  P2 plays tails\\
\end{document}

\subsection{Question 3: Final resulting probabilities}
Below are the final probabilities when run over a number of moves. Each of the probabilities are close to 0.5 and 
as the number of trials/moves increases, we can see that it converges even closer to this probability. Starting strategy (a, b) represents the number of heads each player played on their first go. So, (1, 0) would mean that player 1 played heads on their first go and player 2 played tails.
\begin{table}
	\begin{tabular}{|p{3cm}| p{3cm} | p{6cm}|}
	\hline
	Number of moves & Starting strategy & Final probability ((p1), (p2)) \\
	\hline
	50 & (0, 0) & ((0.44, 0.56), (0.56, 0.44)) \\
	\hline
	50 & (1, 0) & ((0.52, 0.479), (0.44, 0.56)) \\
	\hline
	50 & (0, 1) & ((0.56, 0.44), (0.479, 0.521)) \\
	\hline
	50 & (1, 1) & ((0.54, 0.46), (0.46, 0.54)) \\
	\hline
	100 & (0, 0) & ((0.469999, 0.53), (0.55, 0.45)) \\
	\hline
	100 & (1, 0) & ((0.51, 0.489), (0.45, 0.55)) \\
	\hline
	100 & (0, 1) & ((0.53, 0.469), (0.469, 0.53)) \\
	\hline
	100 & (1, 1) & ((0.5200, 0.479), (0.46, 0.54)) \\
	\hline
	1000 & (0, 0) & ((0.495, 0.505), (0.483, 0.517)) \\
	\hline
	1000 & (1, 0) & ((0.506, 0.493999), (0.517, 0.4829)) \\
	\hline
	1000 & (0, 1) & ((0.504, 0.496), (0.519, 0.4809)) \\
	\hline
	1000 & (1, 1) & ((0.505, 0.495), (0.518, 0.481999)) \\
	\hline
	10000 & (0, 0) & ((0.4934999, 0.506499), (0.4995, 0.500499)) \\
	\hline
	10000 & (1, 0) & ((0.5066, 0.4934), (0.500499, 0.4995)) \\
	\hline
	10000 & (0, 1) & ((0.506399, 0.493599), (0.5007, 0.4993)) \\
	\hline
	10000 & (1, 1) & ((0.506499, 0.493499), (0.5006, 0.4994)) \\
	\hline
	\end{tabular}
\end{table}