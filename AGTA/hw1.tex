\documentclass[11pt]{report}

\newcommand{\define}[2] {
  \textbf{Definition: #1}
  \begin{center} #2
\end{center}
}

\begin{document}

\title{AGTA HW1 \\ Ashley Peacock \\ S0936300}

\maketitle

\section{Question 1}

Given the matrix: 

\begin{center}
$ \bordermatrix{~ & ~    & ~    & ~ & ~    \cr 
				~ & (6, 5) & (4, 8) & (6, 4) & (9, 2) \cr 
				~ & (4, 6) & (7, 4) & (7, 5) & (4, 4) \cr
				~ & (4, 7) & (4, 4) & (9, 5) & (2, 6) \cr
				~ & (5, 9) & (4, 10)& (4, 9) & (8, 9) \cr}$ 
\end{center}
				
To compute Nash equilibrium we start by removing dominated strategies.

\begin{enumerate}
\item Column 4 is weakly dominated by Column 1.
\item Column 3 is weakly dominated by Column 1.
\item Row 4 is is weakly dominated by Row 1.
\end{enumerate}

Removing the dominated strategies we are left with:

\begin{center}
$\bordermatrix{~ & ~      & ~  \cr
			   ~ & (6, 5) & (4, 8) \cr
			   ~ & (4, 6) & (7, 4) \cr
			   ~ & (4, 7) & (4, 4) \cr} $
\end{center}
			   
Row 3 is dominated by row 1 and 2. This leaves us with the following game:

\begin{center}
$\bordermatrix{~ & A      & B  \cr
			   C & (6, 5) & (4, 8) \cr
			   D & (4, 6) & (7, 4) \cr} $
\end{center}

There is no pure strategy nash equilibrium. The game cannot be reduced to a single strategy, i.e there is no strict dominant strategy for both players. Nash's theorem however states that every finite game does have a mixed nash equilibrium.

Let $q$ be the probability that player 2 plays strategy A and $(1 - q)$ the probability player 2 plays strategy B.
Let $p$ be the probability that player 1 plays strategy C and $(1 - p)$ the probability player 1 plays strategy D.

Pay off for player 1:
\begin{center}
$6q + 4(1 - q) = 4q + 7(1 - q)$ \\
$13q = 8q + 3$\\
$q = 3/5$\\
\end{center}

Pay off for player 2
\begin{center}
$5p + 6(1 - p) = 8p + 4(1 - p)$\\
$-p + 6 = 4p + 4$\\
$5p = 2$\\
$p = 2 / 5$
\end{center}

Profiles:\\
$x_{1} = (2/5, 3/5 0, 0)$ \\
$x_{2} = (3/5, 2/5, 0, 0)$ \\

\section{Question 2}



\end{document}