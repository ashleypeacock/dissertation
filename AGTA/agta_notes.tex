\documentclass[11pt]{report}

\newcommand{\define}[2] {
  \textbf{Definition: #1}
  \begin{center} #2
\end{center}
}

\begin{document}

\title{AGTA notes}

\maketitle
\tableofcontents

\chapter{Mixed and pure strategies}


\chapter{Nash equilibrium}
Nash equilibrium occurs when two players are simultaneously playing their best strategy and no player can benefit by deviating from this strategy. Every game has a mixed strategy Nash equilibrium but not all games have a pure strategy Nash.

\section{Methods}

\section{Informal methods}
The following is a quick, informal method of finding Nash equilibrium although it is not advised to use definitively.

\subsection*{Example}

Consider the following matrix:

\begin{center}
$ \bordermatrix{~ & A    & B    & C    \cr 
				D & 5, 1 & 2, 0 & 2, 2 \cr 
				E & 0, 4 & 1, 5 & 4, 5 \cr
				F & 2, 3 & 3, 6 & 1, 0 \cr}$ \end{center}  
				
In order to find the Nash Equilibrium, we need to find each players best responses to each of the other players best responses. Starting with player 1 (the rows), we highlight the best response that player 1 can make if player 2 plays A:

\begin{center}
$ \bordermatrix{~ & A & B & C \cr 
				D & \textbf{5}, 1 & 2, 0 & 2, 2 \cr 
				E & 0, 4 & 1, 5 & 4, 5 \cr
				F & 2, 3 & 3, 6 & 1, 0 \cr}$  
				\end{center}

And then player 1's best response if player 2 plays B or C:
\begin{center}
$ \bordermatrix{~ & A & B & C \cr 
				D & \textbf{5}, 1 & 2, 0 & 2, 2 \cr 
				E & 0, 4 & 1, 5 & \textbf{4}, 5 \cr
				F & 2, 3 & \textbf{3}, 6 & 1, 0 \cr}$  
\end{center}		
We then do the same for player 2 for responses to player 1, obtaining.
\begin{center}
$ \bordermatrix{~ & A & B & C \cr 
				D & \textbf{5}, 1 & 2, 0 & 2, \textbf{2} \cr 
				E & 0, 4 & 1, \textbf{5} & \textbf{4}, \textbf{5} \cr
				F & 2, 3 & \textbf{3}, \textbf{6} & 1, 0 \cr}$  
\end{center}		
As (4, 5) and (3, 6) are both underlined, these are the Nash equilibriums of the game.
			   

\section{Proofs and properties}

\chapter{Minimax and Maximin}
For 2-player zero-sum games, Nash Equilibria and Minimax profiles are the same thing. 

\chapter{Iterative Dominance}
We can intuitively see that if one strategy is better than another(i.e it is a dominant strategy) than the dominated strategy will never be played. The order that iterative dominance is conducted does not effect the result.

There are two types types of dominance - strict and weak dominance.

If both player are playing strictly dominated strategies, it must be a unique Nash Equilibrium. 

\section*{Strictly dominated strategy}
\begin{quote}
A strictly dominated strategy can never be a best reply.
\end{quote}
Thus, we should remove it as it will never be played. 

If both player are playing strictly dominated strategies, it must be a unique Nash Equilibrium. 

\chapter{Normal form games}

\section{Perfect information games}

\section{Imperfect information games}
An imperfect-information game is defined similarly to the Perfect information game, except that it has the element $I$ to represent equivalence classes. I is the set of equivilence classes where $I = (I_{i, j}, ..., I_{i, k})$

p(h) = p(h`): belong to the same player. 

How should we define the pure strategies for each player? Pure strategies are the cross product of the equivalence classes and the set of actions. Thus the result is there are less pure strategies then perfect information games. 

\chapter{Notation and Definitions }

\section{Basic definitions}
\define{Mixed strategy profile}{A set of all the possible combinations of mixed strategies usually denoted by $x_{i}$. A mixed strategy is pure if $x_{i} = 1$ and this is denoted $pi_{i, j}$}
\define{Pure strategy profile}{A set of all the possible combinations of mixed strategies usually denoted by $S_{i}$}
\define{Mixed strategy}{ A mixed strategy is a randomised strategy $x_i$ with a probability distribution over $S_i$(\textbf{pure strategies}). A player will choose a random strategy based on the probabilities of $x_{i}$. In other words $x_{i}$ is a vector of probabilities that sum: \\ $x_{i}(1) + x_{i}(2) + x_{i}(3) + .... + x_{i}(m) = 1$ }
\define{Zero-sum game}{In a zero sum game the utilities/pay-offs of the players must sum to zero. \\ In other words $u_{1} = -u_{2}$}
\define{$u_{i}(a_{i}, a_{-i})$}{the pay off for playing $a_{i}$, regardless of all other strategies}
\define{$u_{i}(a_{i}, a_{-i}) < u_{i}(a`_{i}, a_{-i})$}{utility of $a`_{i}$ is better than $a_{i}$}
\define{Pareto efficient}{To do}
\define{ESS: Evolutionary Stable Solution}{todo}
\define{Perfect information game}{An extensive form game G is called a game of perfect information if every information set $Info_{i, j}$}{has only one node}
\define{Value of a game}{The value of a game is the minimum expected loss/game obtained by paying. The procedure each uses to insure this return is called an optimal strategy or minimax strategy}

\section{Rules}

\define{Single nash equilibrium}{If a matrix can be reduced by strict dominance to a single strategy, this strategy is the only pure nash equilibrium of the game}


\end{document}