\documentclass[11pt]{article}
\begin{document}

\title{AGTA notes}

\maketitle

\section{Nash equilibrium}
Nash equilibrium occurs when two players are simultaneously playing their best strategy and no player can benefit by deviating from this strategy. Every game has a mixed strategy Nash equilibrium but not all games have a pure strategy Nash.

\subsection{Methods}

\subsection{Informal methods}
The following is a quick, informal method of finding Nash equilibrium although it is not advised to use definitively.

\subsubsection*{Example}

Consider the following matrix:

\begin{center}
$ \bordermatrix{~ & A    & B    & C    \cr 
				D & 5, 1 & 2, 0 & 2, 2 \cr 
				E & 0, 4 & 1, 5 & 4, 5 \cr
				F & 2, 3 & 3, 6 & 1, 0 \cr}$ \end{center}  
				
In order to find the Nash Equilibrium, we need to find each players best responses to each of the other players best responses. Starting with player 1 (the rows), we highlight the best response that player 1 can make if player 2 plays A:

\begin{center}
$ \bordermatrix{~ & A & B & C \cr 
				D & \textbf{5}, 1 & 2, 0 & 2, 2 \cr 
				E & 0, 4 & 1, 5 & 4, 5 \cr
				F & 2, 3 & 3, 6 & 1, 0 \cr}$  
				\end{center}

And then player 1's best response if player 2 plays B or C:
\begin{center}
$ \bordermatrix{~ & A & B & C \cr 
				D & \textbf{5}, 1 & 2, 0 & 2, 2 \cr 
				E & 0, 4 & 1, 5 & \textbf{4}, 5 \cr
				F & 2, 3 & \textbf{3}, 6 & 1, 0 \cr}$  
\end{center}		
We then do the same for player 2 for responses to player 1, obtaining.
\begin{center}
$ \bordermatrix{~ & A & B & C \cr 
				D & \textbf{5}, 1 & 2, 0 & 2, \textbf{2} \cr 
				E & 0, 4 & 1, \textbf{5} & \textbf{4}, \textbf{5} \cr
				F & 2, 3 & \textbf{3}, \textbf{6} & 1, 0 \cr}$  
\end{center}		
As (4, 5) and (3, 6) are both underlined, these are the Nash equilibriums of the game.
			   

\subsection{Proofs and properties}

\section{Iterative Dominance}
We can intuitively see that if one strategy is better than another(i.e it is a dominant strategy) than the dominated strategy will never be played. The order that iterative dominance is conducted does not effect the result.

There are two types types of dominance - strict and weak dominance.

If a game can be reduced to one strategy by strict dominance then we can say that there is only one Nash Equilibrium of the game. 

\subsection*{Strictly dominated strategy}
\begin{quote}
A strictly dominated strategy can never be a best reply.
\end{quote}
Thus, we should remove it as it will never be played. 

\section{Notation and Definitions }
$u_{i}(a_{i}, a_{-i})$: the pay off for playing $a_{i}$, regardless of all other strategies \\
$u_{i}(a_{i}, a_{-i}) < u_{i}(a`_{i}, a_{-i})$: utility of $a`_{i}$ is better than $a_{i}$

\section{Examples}
\subsection{Nash Equilibrium}

\end{document}