\documentclass[11pt]{article}

\newcommand{\define}[2] {
  \textbf{Definition: #1}
  \begin{center} #2
\end{center}
}

\begin{document}

\section{Lecture 1: Introduction}

There is a diversity of methods currently used in Neuroscience:

\begin{enumerate}
\item Psychophysics
\item EEG/ERP
\item MEG
\item MRI/fMRI
\item Single neuron recordings and multiple neuron recordings
\end{enumerate}

We can also consider whether methods are invasive(directly interferers with neurons\cite{check}) or non-invasive. 

\subsection*{Psychophysics}
Psychophysics is a sub-discipline of psychology dealing with the relationship between physical stimuli and their perception. It is interested in measuring thresholds of perception, detection, discrimination. Measures illusions, reaction times, effects of training, group differences, effect of substance intake etc. It is non-invasive.  

\subsection*{Models}
A tool of neuroscience is mathmatical and computer models which are used to understand how the brain works. The aims are 

\begin{enumerate}
\item What? Description: unify data in a single framework
\item How? Understand mechanisms
\item Why? Interpretive model: Understand principles underlying functions
\item Make predictions - guide experiments and better data analysis
\end{enumerate}

\section{Glossary}
\define{Psychophysics}{Investigates the relationship between physical stimuli and sensations and the perceptions they affect}

\end{document}